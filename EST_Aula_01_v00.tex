\documentclass[graphics,14pt]{beamer}
\geometry{papersize={338.7mm,190.5mm},margin={1.125cm,0.125cm}}
\usepackage[utf8]{inputenc}
\usepackage[brazil]{babel}
\usetheme{boxes}
\usepackage[sfdefault]{carlito}
\usepackage{natbib}
% -----------------------------------------------------------------------
% --- Slide Justificado -------------------------------------------------
\usepackage{ragged2e}
\usepackage{etoolbox}
\usepackage{scrextend} % ajustar texto para direita
\usepackage{soulutf8}
\usepackage[normalem]{ulem}
\usepackage{cancel}
\usepackage{bm}
\usepackage{array}
\usepackage{wrapfig} 
\apptocmd{\frame}{}{\justifying}{} % Allow optional arguments after frame.
\usepackage{amsmath,mathtools}
\usepackage{verbatim}
\usepackage{tabularx}
% -----------------------------------------------------------------------
% separação entre colunas e linhas de tabelas
\setlength{\tabcolsep}{0.5cm}
\renewcommand{\arraystretch}{1}
% -----------------------------------------------------------------------
% --- Define a centralização de coluna com largura {|C{1 cm}|}
\newcolumntype{C}[1]{>{\centering\let\newline\\\arraybackslash\hspace{0pt}}m{#1}}
\newcolumntype{L}[1]{>{\raggedright\let\newline\\\arraybackslash\hspace{0pt}}m{#1}}
% -----------------------------------------------------------------------
\newcommand\litem[1]{\item{\bfseries #1 }}

% --- Letras e simbolos matemáticos -------------------------------------
%\usepackage[bbgreekl]{mathbbol}  % Pacote para repreentação de conjuntos com \mathbb{R} com letras gregas
%\usepackage{amsfonts}	% Pacote para repreentação de conjuntos com \mathbb{R}
%\usepackage{mathrsfs}	% Pacote para letras matemáticas
%\usepackage{amssymb} 	% diversos simbolos matematicos adicionais. Carrega automático com amsfonts
% ------------------------------------------------------------------------
\usepackage{multicol}
\usepackage{multimedia}
\usepackage[]{graphicx}
\usepackage[]{color}
\usepackage{geometry}
%\usepackage{media9}

\usepackage{tabularx}
%\usepackage{amsmath, amsthm, amssymb}
\usepackage{gensymb}

% ------------------------------------------------------------------------

% --- Esta definicao deve vir antes --------------------------------------
\usepackage{graphicx}			% Inclusão de gráficos
\usepackage{float}  
% ------------------------------------------------------------------------
\usefonttheme[onlymath]{serif}
\usepackage{hyperref}
 \usepackage{breakurl}
\usepackage{multirow}
\usepackage{subfig}
\usepackage{ragged2e}
%\captionsetup[subfigure]{labelformat=empty}
\usepackage{color}
\usepackage{colortbl}
\usepackage{textpos}
\usepackage{tikz}
\usetikzlibrary{calc}
%---------------
\usepackage{tabularx}
\usepackage{booktabs}
\usepackage{multimedia}
\usepackage[style=british]{csquotes}
\usepackage{listings}
%\usepackage[absolute,overlay]{textpos}
% ------------------------------------------------------------------------
\lstset{language=R,
	basicstyle=\large\ttfamily,
	stringstyle=\color{teal},
	otherkeywords={0,1,2,3,4,5,6,7,8,9},
	morekeywords={TRUE,FALSE},
	deletekeywords={data,frame,length,as,character},
	keywordstyle=\color{blue},
	commentstyle=\color{teal},
	showspaces=false,               % show spaces adding particular underscores
	showstringspaces=false,         % underline spaces within strings
	showtabs=false,                 % show tabs within strings adding particular underscores
%	frame=single,                   % adds a frame around the code
	tabsize=1,                      % sets default tabsize to 2 spaces
	captionpos=b,                   % sets the caption-position to bottom
	breaklines=true,                % sets automatic line breaking
	breakatwhitespace=false, 
}
% ------------------------------------------------------------------------
\pdfsuppresswarningpagegroup=1
\def\disciplina/{Estatística para Linguística}
% ------------------------------------------------------------------------
\tikzset{
	invisible/.style={opacity=0},
	visible on/.style={alt={#1{}{invisible}}},
	alt/.code args={<#1>#2#3}{%
		\alt<#1>{\pgfkeysalso{#2}}{\pgfkeysalso{#3}} % \pgfkeysalso doesn't change the path
	},
}
% ------------------------------------------------------------------------
\def\signed #1{{\leavevmode\unskip\nobreak\hfil\penalty50\hskip1em
		\hbox{}\nobreak\hfill #1%
		\parfillskip=0pt \finalhyphendemerits=0 \endgraf}}
\newsavebox\mybox
\newenvironment{aquote}[1]
{\savebox\mybox{#1}\begin{quote}}
	{\vspace*{1mm}\signed{\usebox\mybox}\end{quote}}
% ------------------------------------------------------------------------
% --- Desativando os botoes de navegacao ---------------------------------
\setbeamertemplate{navigation symbols}{}
% ------------------------------------------------------------------------
% --- Tela cheia ---------------------------------------------------------
\hypersetup{pdfpagemode=FullScreen}
% ------------------------------------------------------------------------
% --- Layout da pagina ---------------------------------------------------
\hypersetup{pdfpagelayout=SinglePage}
% ------------------------------------------------------------------------
% --- Relaxed footnotes --------------------------------------------------
\newcommand{\lfr}[1]{\let\thefootnote\relax\footnote{\hspace{0.6cm}\vspace{1.25cm} #1}}
% --- Pasta com as imagens -----------------------------------------------
\graphicspath{{Imagens/}}
% --- Define o cinza tema ------------------------------------------------
\definecolor{pcmggray}{rgb}{0.5, 0.5, 0.5}
% --- Define o estilo do titulo do frame ---------------------------------
\setbeamertemplate{frametitle}{ 
	\Huge{\bfseries{\insertframetitle\par}\vskip-18pt\hrulefill}
	\begin{tikzpicture}[remember picture,overlay]
	\node[xshift=0.65cm,yshift=-1.28 cm,opacity=1.0] at (current page.north west) {\includegraphics[width=0.56cm]{00BAS_marcador_02.pdf}};
	\end{tikzpicture}
}
%\multicolbaselineskip = 1cm
 \columnsep = 1cm
% ------------------------------------------------------------------------
\setbeamercolor{frametitle}{fg=pcmggray,bg=white}
\setbeamerfont{structure}{size=\LARGE}
\setbeamerfont{itemize/enumerate body}{size=\LARGE}
\setbeamerfont{itemize/enumerate subbody}{size=\LARGE}
\setbeamerfont{normal text}{size=\LARGE}
\setbeamercolor{structure}{fg=black} % itemize, enumerate, etc
\setbeamercolor{section in toc}{fg=black} % TOC sections
\setbeamertemplate{bibliography item}{}
\setbeamerfont{bibliography item}{size=\normalsize}
\setbeamerfont{bibliography entry author}{size=\normalsize}
\setbeamerfont{bibliography entry title}{size=\normalsize,series=\bfseries}
\setbeamerfont{bibliography entry location}{size=\normalsize}
\setbeamerfont{bibliography entry note}{size=\normalsize}
\AtBeginDocument{\usebeamerfont{normal text}}
% --- Fundo da pagina de titulo ------------------------------------------
\setbeamertemplate{background} 
{
	\begin{tikzpicture}[remember picture, overlay]
	\node[xshift=5.75cm,yshift=-1.68cm,opacity=1.0] at (current page.north west) {\includegraphics[width=11.49cm]{00BAS_titleleftupimage.pdf}};
	\node[xshift=16.93cm,yshift=-6.47cm,opacity=1.0] at (current page.north west) {\includegraphics[width=18.0cm]{00BAS_ada_ufmg.pdf}};
	\node[xshift=28.35cm,yshift=-17.48cm,opacity=1.0] at (current page.north west) {\includegraphics[width=11.04cm]{00BAS_titlerightdown.pdf}};
	\end{tikzpicture}
}
% ------------------------------------------------------------------------
\defbeamertemplate*{title page}{customized}[1][]
{
	\begin{center}
	\vspace{-1.5cm}
	\usebeamerfont{title}\LARGE{\textbf{\inserttitle}}\par
	\vspace{9cm}
	\usebeamerfont{subtitle}\huge{\textbf{\insertsubtitle}}\par
	\vspace{2cm}
	\usebeamerfont{author}\textbf{\insertauthor}\par
	\end{center}
	\vfill
}
\title{Laboratório de Fonologia}
\subtitle{\disciplina/}
\author[Silva, A. P.]{Prof. Dr. Adelino Pinheiro Silva}
\date{\today}
% =======================================================================
% INICIO DO CONTEUDO PRINCIPAL
% =======================================================================
\begin{document}
 \frame{\titlepage}
% =======================================================================
% ---- Fundo da demais paginas ------------------------------------------
\setbeamertemplate{background} 
{
	\begin{tikzpicture}[remember picture, overlay]
	\node[xshift=1.14cm,yshift=-18.08cm,opacity=1.0] at (current page.north west) {\includegraphics[width=1.58cm]{00BAS_fale_logo.pdf}};
	\node[xshift=28.97cm,yshift=-18.2cm,opacity=1.0] at (current page.north west) {\includegraphics[width=9.8cm]{00BAS_regularrightdown.pdf}};
	\end{tikzpicture}
	\put(880,-20){\Large{\insertframenumber \hspace{2pt} de \inserttotalframenumber}}
%	\begin{textblock}{2cm}(150pt,1pt)
%		\normalsize \insertframenumber de \inserttotalframenumber
%	\end{textblock}
%	\normalsize \insertframenumber de \inserttotalframenumber
}
% ==== Sumário ==========================================================
\section{Introdução}
\begin{frame}[t]
    \frametitle{Sumário}
    \begin{multicols}{2}
	    \tableofcontents
    \end{multicols}

\end{frame}
% ========================================================================

% ========================================================================
\AtBeginSection[]
{
  \begin{frame}[t]
    \frametitle{Assunto}
    \begin{multicols}{2}
    	\tableofcontents[currentsection]
    \end{multicols}
  \end{frame}
}
% =======================================================================

% ==== AULA 01 ==========================================================
\section{Introdução}
% ------------------------------------------------------------------------
\begin{frame}
	\frametitle{In a hole in the ground there lived a...}

	Por que estudar estatística?
	\begin{itemize}
		\item[-] Compreender \textbf{fatores} que afetam um resultado.
		\item[-] Julgar de forma crítica as informações recebidas.
		\item[-] Argumentar estatisticamente.
	\end{itemize}
	
	\vspace{2cm}
		
	O que é estatística \citep{Agresti2018}?
	\begin{itemize}
		\item[-] Conjunto de métodos para se \textbf{obter} e \textbf{analisar} dados.
		\item[-] Metodologia baseada na \textbf{ocorrência} \\ para realizar \textbf{previsão}.
	\end{itemize}

	\begin{tikzpicture}[remember picture,overlay]
		\node[xshift=10.5cm,yshift=1.50cm,opacity=1.0] at (current page.center) {\includegraphics[width=12cm]{est_01_desvio_padrao.jpg}};
	\end{tikzpicture}
	\vfill
	\lfr{Imagem: \url{https://i.pinimg.com/originals/48/fb/95/48fb9570a8413fe83053df7c3599e7eb.jpg}}
\end{frame} 
% ------------------------------------------------------------------------
\begin{frame}[t]
\frametitle{In a hole in the ground there lived a...}

\begin{aquote}{}
	``Acho que somos forçados a concluir que a gramática é autônoma e independente do significado, e que os modelos probabilísticos não fornecem nenhum entendimento particular dentro de alguns problemas básicos da estrutura sintática. (\textbf{tradução minha})'' \cite[p.-17]{Chomsky2009} citado em \cite[p.~2]{Levshina2015}
\end{aquote}

\vspace{2cm}

O que é estatística não pode fazer \citep{Levshina2015}
\begin{itemize}
	\item[-] O \textit{software} estatístico não pode fazer a pesquisa por você. 
	\item[-] As estatísticas não respondem o ``por quê''. 
	\item[-] A causalidade é sempre imposta pelo pesquisador com base em suas considerações teóricas, dados empíricos e senso comum.
\end{itemize}

\end{frame} 
% ------------------------------------------------------------------------
\begin{frame}
	\frametitle{I Have the High Ground}

	Alguns termos para começar
	\begin{itemize}
		\item[-] \textbf{Dado}: Observação obtida sobre o objeto de interesse.
		\item[-] \textbf{Observação}: Medida, ou informação coletada (sujeita a ruído e erros).
		\item[-] \textbf{Base de dados}: Conjunto de dados, e.g., \textit{general social survey}.
		\item[-] \textbf{População}: Conjunto total dos elementos (desconhecido, inacessível).
		\item[-] \textbf{Amostra}: subconjunto da população, dados (medidas) coletados.
		\item[-] \textbf{Parâmetro}: Fator (resumo) numérico da população (dica: letras gregas).
		\item[-] \textbf{Estatística}: Valor obtido da amostra !!!!!
		\item[-] \textbf{Ferramental}: R-studio
	\end{itemize}
	\begin{tikzpicture}[remember picture,overlay]
	\node[xshift=7.5cm,yshift=-4.0cm,opacity=1.0] at (current page.center) {\includegraphics[width=12cm]{R-studio-logo.png}};
	\end{tikzpicture}
	\vfill
	\lfr{Imagem: \url{https://www.python.org/static/community_logos/python-logo-generic.svg}}
\end{frame}  
% ------------------------------------------------------------------------
\begin{frame}
\frametitle{I Have the High Ground}
	\begin{tikzpicture}[remember picture,overlay]
	\node[xshift=0cm,yshift=0.0cm,opacity=1.0] at (current page.center) {\includegraphics[width=22cm]{amostra-populacao_v0.png}};
	\end{tikzpicture}
	\vfill
	\lfr{Imagem: \url{https://fernandafperes.com.br/intervalo-de-confianca/}}
\end{frame}
% ------------------------------------------------------------------------
\begin{frame}
\frametitle{Medida e amostra}
	Maneiras de extrair informações de interesse.\\
	\begin{itemize}
		\item[-] \textbf{Variável aleatória}: Característica que pode variar com os elementos da população ou amostra.
		
		\item[-] \textbf{Escala de medição}: Extensão onde a variável aleatória pode ser medida. Exemplos:
			\begin{itemize}
				\item[-] Categóricas: (cara, coroa), (derrota, empate, vitória); ou
				\item[-] Quantitativas: $\{ x \in \mathbb{R} | 0 \leq x \leq 1\} $, $[0,1]$
			\end{itemize}
	\end{itemize}


	Se caracteriza a variável aleatória como um resultado de uma experiência aleatória, que pode ser classificada como:
	\begin{itemize}
		\item[-] \textbf{Categóricas}: valores aceitos dentro de um limite de categorias (qualitativos?).
		\item[-] \textbf{Quantitativas}: valores numéricos de qualquer conjunto, e.g., $\mathbb{N}$, $\mathbb{R}$, $\mathbb{C}$
	\end{itemize}
	\vfill
	\lfr{Definição formal de variável aleatória: $X:\Omega \to \mathbb{R} \Leftrightarrow \{ \omega: X(\omega) \leq x \} \in \mathcal{F}, \forall x \in \mathbb{R} $}
\end{frame}

% ------------------------------------------------------------------------
\begin{frame}
\frametitle{Medida e amostra}
	Escalas:.\\
	\begin{itemize}
		\item[-] \textbf{Intervalar}: delimitação numérica.
		\item[-] \textbf{Nominal}: Nomes/categorias ``não ordenáveis'', e.g., preferência de cores;
		\item[-] \textbf{Ordenáveis}: Nomes/categorias que podem ser ordenadas em níveis, e.g., expectativa do curso (baixá, sem expectativa, alta).	
	\end{itemize}
	
	Detalhe: Em escalas categóricas é muito difícil garantir uma homogeneidade dos intervalos, i.e., se os intervalos das categorias possuem escalas de mesmo tamanho.
	
	

\end{frame}
% ------------------------------------------------------------------------
\begin{frame}
\frametitle{Variáveis estatísticas}
	\begin{columns}[T] % align columns
		\begin{column}{.40\textwidth}
			\begin{itemize}
				\item[-] \textbf{Amostra aleatória simples}: todas amostras de mesmo tamanho possem a mesma ``chance''. Seria um retrato da população(?).
				\item[-] \textbf{Métodos de amostragem, \textit{sample survey}}: Sistemática, estratificada, grupo (\textit{cluster}), multiestágios.
				\item[-] \textbf{Amostra enviesada}: alunos de uma sala de aula (?).
			\end{itemize}
		\end{column}%
		\hfill%
		\begin{column}{.48\textwidth}	
		
		\end{column}%
	\end{columns}
	\begin{tikzpicture}[remember picture,overlay]
	\node[xshift=7cm,yshift=0.0cm,opacity=1.0] at (current page.center) {\includegraphics[width=18cm]{variaveis-estatisticas.png}};
	\end{tikzpicture}
	\vfill
	\lfr{Adaptado de \url{https://artemetabolica.blogspot.com/2015/12/bioestatistica-1-variaveis.html}}
\end{frame}
% ------------------------------------------------------------------------
\begin{frame}
\frametitle{Estudo experimental}
	\textbf{Experimento:} Controlar variáveis independentes e observar a variação de variáveis dependentes para dar suporte ou refutar uma hipótese.\\
	\begin{itemize}
		\item[-] Compara ``tratamentos''.
		\item[-] Unidades de testes.
		\item[-] Grupos, pelo menos, ``controle'' e ``tratamento''.
		\item[-] Variáveis estranhas (predatórias).
	\end{itemize}
%	Efeitos do teste: principal e interativo\\
%	Regressão analítica
	Problemas experimentais\\
		\begin{itemize}
		\item[-] Variação do instrumento (ou pessoa que conduz parte dele).
		\item[-] Regressão analítica.
		\item[-] Viés de seleção.
		\item[-] Perda de unidade
	\end{itemize}	
	\begin{tikzpicture}[remember picture,overlay]
	\node[xshift=12cm,yshift=-4.0cm,opacity=1.0] at (current page.center) {\includegraphics[width=10cm]{Experimento-criancas.jpg}};
	\end{tikzpicture}

	\vfill
	\lfr{ \url{https://br.freepik.com/vetores-premium/}}
\end{frame}
% ------------------------------------------------------------------------
\begin{frame}
\frametitle{Estudo experimental}
	\vspace{-1.5cm}
	Efeitos do teste: principal e interativo\\
	\vspace{1cm}
	\textbf{Soluções para experimentos}:\\
	\begin{itemize}
		\item[-] Aleatorização.
		\item[-] Emparelhamento.
		\item[-] controle estatístico.
		\item[-] Planejamento.
		\item[-] Medições a \textit{posteriori}.
	\end{itemize}
	\begin{tikzpicture}[remember picture,overlay]
	\node[xshift=12cm,yshift=-4.0cm,opacity=1.0] at (current page.center) {\includegraphics[width=10cm]{Experimento-criancas.jpg}};
	\end{tikzpicture}
	
	\vfill
	\lfr{ \url{https://br.freepik.com/vetores-premium/}}
	%	Efeitos do teste: pr
\end{frame}
% ------------------------------------------------------------------------
\begin{frame}
\frametitle{Estudo de Observação}
	\begin{tikzpicture}[remember picture,overlay]
\node[xshift=5cm,yshift=-2.5cm,opacity=1.0] at (current page.center) {\includegraphics[width=23cm]{exemplo-de-correlacao-sem-causalidade.png}};
\end{tikzpicture}
\vspace{-2.5cm}
	\begin{itemize}
		\item[-] Sem manipulação do objeto de estudo.
		\item[-] Grupos desbalanceados, difícil de realizar uma comparação adequada.
		\item[-] \textbf{Não permite estabelecer causa e efeito}.
		\item[-] Pode indicar uma relação \\ entre variáveis.
		\item[-] Uma variável não medida \\ pode ser responsável \\ pelo padrão observado.
	\end{itemize}
%	\begin{tikzpicture}[remember picture,overlay]
%	\node[xshift=12cm,yshift=-4.0cm,opacity=1.0] at (current page.center) {\includegraphics[width=12cm]{exemplo-de-correlação-sem-causalidade.jpg}};
%	\end{tikzpicture}

	\vfill
	\lfr{ \url{http://blog.bravi.com.br/wp-content/uploads/2017/10}}
\end{frame}
% ------------------------------------------------------------------------
\begin{frame}
\frametitle{Variabilidade amostral e viés}
\vspace{-2cm}
	\textbf{Erro de amostragem:} erro ocorrido ao utilizar uma estatística da amostra para predizer um parâmetro da população. Exemplo: Erro da pesquisa eleitoral com n = 1000 de + ou - 3\%. \\
	\vspace{1cm}
	Viés: erro quando a amostra é enviesada, e.g., voluntários ou respostas de carta.\\
	\begin{itemize}
		\item[-] \textbf{Viés de resposta} ocorre quando a pergunta é confusa, e.g., referendo do desarmamento;
		\item[-] \textbf{viés de falha de dados} apenas uma fatia da amostra responde.
	\end{itemize}
	\vfill
\end{frame}
% ------------------------------------------------------------------------
	\begin{frame}
	\frametitle{Fim da introdução - Dever de casa}
		\textbf{Exercícios do livro \cite{Agresti2018}:}
		\begin{itemize}
			\item[-] Capítulo 1: 1.1, 1.3, 1.5-1.8, 1.14, 1.16;
			\item[-] Capítulo 2: 2.2-2.10,2.27, 2.35-2.37,2.39
		\end{itemize}
		\textbf{Preparação do terreno}
		\begin{itemize}
			\item[-] Instalar o R-studio.
		\end{itemize}	

	\end{frame}   
% ------------------------------------------------------------------------
\section{Estatística Descritiva}
% ------------------------------------------------------------------------
\begin{frame}[t,fragile=singleslide]
\frametitle{Estatística descritiva}
	Primeiro passo para entender os dados coletados\\
	Facilitar a assimilação de informação\\
	Medidas de:
	\begin{itemize}
		\item[-] tendência central (média), variabilidade, associação, etc...
	\end{itemize}
	Análise e regressão: predizer uma variável a partir de outras.\\
	\vspace{1.0cm}
	Um pouco de código R para tratar com dados
	\begin{lstlisting}
	data_lemas <- read.table("../Dados/lexporbr_alfa_lemas_txt.txt", header = TRUE, 
							sep = "\t",dec = ",",quote = "\"")
	head(data_lemas)
	dim(data_lemas)
	summary(data_lemas)
	\end{lstlisting}
	\vspace{1.0cm}	
	Dados de \href{https://www.lexicodoportugues.com/}{Corpus Léxico do português}
	\vfill
	\lfr{Dados: \url{https://www.lexicodoportugues.com/downloads/lexporbr_alfa_lemas_txt.txt}}

\end{frame}
% ------------------------------------------------------------------------
\begin{frame}[t,fragile=singleslide]
\frametitle{Tabelas e gráficos}
	Extraindo o cabeçalho dos dados
	\begin{lstlisting}
	> head(data_lemas)
	\end{lstlisting}
	Gera a saída:
	\begin{addmargin}[2.45cm]{0em}
		\normalsize 
		\begin{verbatim}
		id ortografia cat_gram inf_gram freq_orto freq_orto.M log10_freq_orto zipf_escala nb_letras
		1  1          o     gram      det   4364416   139093.06          6.6399      8.1433         1
		2  2         de     gram      prp   2553292    81372.90          6.4071      7.9105         2
		3  3          ,     gram       pu   2133025    67979.08          6.3290      7.8324         1
		4  4          .     gram       pu   1603184    51093.15          6.2050      7.7084         1
		5  5         em     gram      prp   1044260    33280.36          6.0188      7.5222         2
		6  6          e     gram       kc    667736    21280.61          5.8246      7.3280         1
		\end{verbatim}
	\end{addmargin}
	A dimensionalidade dos dados, onde cada linha indica uma medição com as colunas indicando as informações\\
	\begin{lstlisting}
	> dim(data_lemas)
	\end{lstlisting}
	Que é um total de 169.606 linhas com 9 colunas
	\begin{addmargin}[2cm]{0em}
	\normalsize 
	\begin{verbatim}
		[1] 169606      9
	\end{verbatim}
	\end{addmargin}

\end{frame}
% ------------------------------------------------------------------------
\begin{frame}[t,fragile=singleslide]
	\frametitle{Tabelas de contingência}
	Construindo uma tabela
	\begin{lstlisting}
	> tab <- table(data_lemas$cat_gram, data_lemas$nb_letras)
	\end{lstlisting}
	\vspace{2.0cm}

	\begin{addmargin}[2cm]{0em}
		\normalsize 
		\begin{verbatim}
		         1     2     3     4     5     6     7     8     9    10    11    12    13
		adj      3    41   355   699  1061  1490  2320  2817  3117  3106  2672  2099  1575
		adv      2    15    31    57    80    96   109   115   148   235   255   313   330
		gram    56    60    96    75    75    47    22    12    14     8     1     2     0
		nom     57   500  1517  3375  4992  5924  7396  7504  7397  6719  5912  4278  3151
		num     11   275  2045  5678 14704 11792  8534  6535  3553   913   941  1630   657
		ver      1    11    51   175  1037  1796  2099  2442  2115  1678  1123   815   475
		\end{verbatim}
	\end{addmargin}
	
\end{frame}
% ------------------------------------------------------------------------
\begin{frame}[t,fragile=singleslide]
\frametitle{Histograma}
	Histograma em uma figura PNG...
	\begin{lstlisting}
	png(file = ""../Imagens/histograma.png",width = 864, height = 486, units = "px")
	hist(data_lemas$nb_letras,main="Histograma do numero de letras em cada ocorrencia",
	breaks=40,xlab = "numero de letras", ylab="Ocorrencias",col = "blue2",
	border="white")
	dev.off()
	\end{lstlisting}
	\begin{tikzpicture}[remember picture,overlay]
	\node[xshift=6cm,yshift=-2.75cm,opacity=1.0] at (current page.center) {\includegraphics[width=24cm]{histograma.png}};
	\end{tikzpicture}
	\vfill
\end{frame}
% ------------------------------------------------------------------------
\begin{frame}[t,fragile=singleslide]
\frametitle{Diagrama de caixa}
	Diagrama de caixa (\textit{boxplot}) em uma figura PNG...
	\begin{lstlisting}
	png(file = "../Imagens/Box_plot.png",width = 864, height = 486, units = "px")
	boxplot(data_lemas$nb_letras ~ cat_gram, data = data_lemas, ylab = "Numero de letras", 
	col = "blue2", border="black")
	dev.off()
	\end{lstlisting}
	\begin{tikzpicture}[remember picture,overlay]
	\node[xshift=4cm,yshift=-2.75cm,opacity=1.0] at (current page.center) {\includegraphics[width=24cm]{Box_plot.png}};
	\end{tikzpicture}
	\vfill
\end{frame}
% ------------------------------------------------------------------------
\begin{frame}[t,fragile=singleslide]
\frametitle{Diagrama ramo e folha}
	Apenas do primeiro ao 300º elemento
	\begin{lstlisting}
	> tab <- stem(data_lemas$nb_letras[1:300])
	\end{lstlisting}
	\vspace{2.0cm}
	
	\begin{addmargin}[2cm]{0em}
		\normalsize 
		\begin{verbatim}
		  The decimal point is at the |
		
		 1 | 0000000000000000000
		 2 | 0000000000000000000
		 3 | 0000000000000000000000000000000000000
		 4 | 00000000000000000000000000000000000000000
		 5 | 00000000000000000000000000000000000000000000000000000000000000
		 6 | 00000000000000000000000000000000000000000000
		 7 | 000000000000000000000000000000000000000
		 8 | 0000000000000000000000
		 9 | 000000000
		10 | 0000000
		11 | 
		12 | 
		13 | 0
		\end{verbatim}
	\end{addmargin}

\end{frame}
% ------------------------------------------------------------------------
\begin{frame}[t,fragile=singleslide]
\frametitle{Medidas de tendência central}
	\begin{tikzpicture}[remember picture,overlay]
	\node[xshift=7cm,yshift=-3.0cm,opacity=1.0] at (current page.center) {\includegraphics[width=18cm]{tendencia-central.png}};
	\end{tikzpicture}
	\vspace{-1.5cm}
	\begin{itemize}
		\item[-] Média
			\begin{itemize}
				\item[-] Aritmética: problema que \textit{outliers} podem alavancar.
				\item[-] Truncada (\textit{winsorized})
				\item[-] Ponderada
			\end{itemize}
		\item[-] Mediana: menos problemas com \textit{outliers}.
		\item[-] Moda: bem indicada para variáveis categóricas.
		\item[-] Tri-média: utiliza quartis.
	\end{itemize}
	
	
	\vfill
	\lfr{Adaptado de: \url{https://analystprep.com/cfa-level-1-exam/wp-content/uploads/2020/01/page-45-1024x673.jpg}}
\end{frame}
% ------------------------------------------------------------------------
\begin{frame}[t,fragile=singleslide]
\frametitle{Medidas de dispersão}
	\begin{tikzpicture}[remember picture,overlay]
	\node[xshift=6cm,yshift=-1.0cm,opacity=1.0] at (current page.center) {\includegraphics[width=18cm]{desvio-sigma.png}};
	\end{tikzpicture}
	\vspace{-1.5cm}
	\begin{itemize}
		\item[-] \textit{Range}, alcance, diferença entre mínimo e máximo.
		\item[-] desvios
		\begin{itemize}
			\item[-] variância e desvio padrão.
			\item[-] Soma dos desvios quadrados.
		\end{itemize}
		\item[-] Distância inter-percentis.
		\item[-] \textbf{Erro padrão}: desvio padrão da média
		\item[-] Regra do $\sigma$: 
		\begin{itemize}
			\item[-] $0,67\sigma \rightarrow 50\%$
			\item[-] $1\sigma \rightarrow 68,3\%$
			\item[-] $1,96\sigma \rightarrow 95\%$
			\item[-] $2\sigma \rightarrow 95,4\%$
			\item[-] $3\sigma \rightarrow 99,7\%$			
		\end{itemize}
	\end{itemize}

	\vfill
	\lfr{Adaptado de: \url{https://civilblog.org/2014/05/11/what-is-standard-deviation-and-how-to-calculate-it-with-an-example-calculation/}}
\end{frame}
% ------------------------------------------------------------------------
\begin{frame}[t,fragile=singleslide]
\frametitle{Achatamento(curtose) e (As)simetria}
	Mais medidas de caracterização dos dados
	\begin{lstlisting}
	library(moments)
	curt_data <- kurtosis(vec_n_letras_sel)
	assi_data <- skewness(vec_n_letras_sel)
	\end{lstlisting}
	\vspace{2.0cm}
	
	\begin{tikzpicture}[remember picture,overlay]
	\node[xshift=-4.5cm,yshift=-3.75cm,opacity=1.0] at (current page.center) {\includegraphics[width=20cm]{different_skewness.png}};
	\end{tikzpicture}
	\begin{tikzpicture}[remember picture,overlay]
	\node[xshift=9.5cm,yshift=2cm,opacity=1.0] at (current page.center) {\includegraphics[width=12cm]{curtose_pdfs.png}};
	\end{tikzpicture}
	\vfill
\lfr{\url{https://en.wikipedia.org/wiki/Kurtosis} e \url{https://en.wikipedia.org/wiki/}}

\end{frame}
% ------------------------------------------------------------------------
\begin{frame}[t,fragile=singleslide]
\frametitle{Exemplos de Estatísticas}
	\begin{tikzpicture}[remember picture,overlay]
	\node[xshift=7.5cm,yshift=-4cm,opacity=1.0] at (current page.center) {\includegraphics[width=18cm]{densidade_plot.png}};
	\end{tikzpicture}
	Extraindo algumas estatísticas dos dados:
	\begin{lstlisting}
	library(moments)
	vec_n_letras_sel <- data_lemas[data_lemas$cat_gram 
				%in% 'gram',]$nb_letras
	data_density <- density(vec_n_letras_sel,n=4096, 
				bw=1.2*bw.nrd(vec_n_letras_sel))
	idx_max <- which.max(data_density$y)
	moda_data <- data_density$y[idx_max]
	mean_data <- mean(vec_n_letras_sel)
	medi_data <- median(vec_n_letras_sel)
	stdv_data <- sd(vec_n_letras_sel)
	curt_data <- kurtosis(vec_n_letras_sel)
	assi_data <- skewness(vec_n_letras_sel)
	\end{lstlisting}
	\vspace{0.5cm}
	\begin{addmargin}[2cm]{0em}
		\normalsize 
		\begin{verbatim}
		3.088693
		4.313808
		4
		2.719188
		8.536706
		1.746169
		\end{verbatim}
	\end{addmargin}

\end{frame}
% ------------------------------------------------------------------------
\begin{frame}[t,fragile=singleslide]
\frametitle{Representações bivariadas}
	\begin{tikzpicture}[remember picture,overlay]
	\node[xshift=8cm,yshift=3.75cm,opacity=1.0] at (current page.center) {\includegraphics[width=16cm]{scatter_plot_02.png}};
	\end{tikzpicture}
	\vspace{-1.5cm}
	\begin{itemize}
		\item[-] Tabelas de contingência
		\item[-] Gráficos de dispersão
		\item[-] Correlação
		\begin{itemize}
			\item[-] Pearson, Kendall, Spearman
		\end{itemize}
		\item[-] Informação mútua...
	\end{itemize}
	\vspace{0.5cm}
	Extraindo dados de duas variáveis:\\
	\begin{lstlisting}
	vec_X_sel <- data_lemas[data_lemas$cat_gram %in% 'gram',]$nb_letras
	vec_Y_sel <- data_lemas[data_lemas$cat_gram %in% 'gram',]$zipf_escala
	png(file = "../Imagens/scatter_plot_02.png",width = 864, height = 486, bg = "transparent")
	plot(x=vec_X_sel,y=vec_Y_sel,type='p',pch=16,xlab="log10_freq_orto",ylab="zipf_escala")
	grid(10,lwd =2)
	dev.off()
	cor(vec_X_sel,vec_Y_sel)
	\end{lstlisting}
	\vspace{0.5cm}
	\begin{addmargin}[2cm]{0em}
		\normalsize 
		\begin{verbatim}
		-0.2453342 
		\end{verbatim}
	\end{addmargin}
\end{frame}
% ------------------------------------------------------------------------
\begin{frame}[t,fragile=singleslide]
\frametitle{Fim da Estatística Descritiva - Dever de casa}
	\textbf{Exercícios do livro \cite{Agresti2018}:}
	\begin{itemize}
		\item[-] Capítulo 3: 1.1, 1.3, 1.5-1.8, 1.14, 1.16;
	\end{itemize}
	\textbf{Preparação do terreno}
	\begin{itemize}
		\item[-] Reproduzir os exemplos no R-Studio.
	\end{itemize}	
	\vspace{1cm}
	Lembrete:\\
	\textbf{Parâmetros} de populações geralmente são representados por letras gregas, e.g., $\mu$ (média), $\sigma^2$ (variância), $\pi$ (proporção), etc...\\
	\textbf{Estatísticas} são extraídas das amostras e representadas por letras latinas, com ou sem complemento, e.g., $m$, $s^2$, $p$.\\
\end{frame}

% ------------------------------------------------------------------------
\section{Probabilidades}
\begin{frame}[t,fragile=singleslide]
\frametitle{Algumas definições}
	\begin{itemize}
		\item[-] Valor que indica o quão suscetível um evento está de ocorrer.
		\item[-] Proporção de um evento em particular dada uma longa sequência de observações
		\item[-] Bases para o cálculo de probabilidades:
		\begin{itemize}
			\item[-] Axiomas de Kolmogorov
			\item[-] Teorema do limite central (ou central do limite?).
			\item[-] $\sigma$-algebra
			\item[-] lei dos grandes números			
		\end{itemize}
		\item[-] Exemplo do problema de Monty Hall.
	\end{itemize}
	\textbf{Parâmetros}: de populações geralmente são representados por letras gregas, e.g., $\mu$ (média), $\sigma^2$ (variância), $\pi$ (proporção), etc...\\
	\textbf{Estatísticas} são extraídas das amostras e representadas por letras latinas, com ou sem complemento, e.g., $m$, $s^2$, $p$.\\
	\begin{tikzpicture}[remember picture,overlay]
	\node[xshift=10cm,yshift=1.75cm,opacity=1.0] at (current page.center) {\includegraphics[width=12cm]{monty-hall.png}};
	\end{tikzpicture}
	\vfill
	\lfr{\url{https://brilliant-staff-media.s3-us-west-2.amazonaws.com/tiffany-wang/gWotbuEdYC.png}}
	
\end{frame}

% ------------------------------------------------------------------------
\begin{frame}[t,fragile=singleslide]
\frametitle{Notação e regras básicas I, mais em \citep{Halperin1965}}
	\begin{columns}[T] % align columns
	\begin{column}{.85\textwidth}
		A variável aleatória $x \in X$ significa que um resultado particular (amostra) $x$ pertence $\in$ a variável aleatória/conjunto (população) $X$.\\
		\vspace{0.5cm}
		Se a variável tem seus parâmetros conhecidos, por exemplo, vem de uma distribuição normal (Gaussiana $\mathcal{N}(\mu,\sigma)$) com média igual a 1,7 e desvio padrão de 0,4 podemos escrever $x \in \mathcal{N}(1,7,0,4)$, ou $X \backsim \mathcal{N}(1,7,0,4)$. \\
		\vspace{0.5cm}
		A normal padrão possui média igual a zero e desvio unitário $\mathcal{N}(0,1)$\\		
		\vspace{0.5cm}
		Uma probabilidade de um evento $A$ é definida como $P(\omega:X(\omega) \in A)$ ou simplesmente $P(A)$ (vide nota de rodapé).
		
	\end{column}
	\hfill
	\begin{column}{0.05\textwidth}
	\end{column}		
	\end{columns}
	\vfill
	\lfr{Definições mais formalizadas, como \cite{Halperin1965} e \cite{Wikipedia2017}, sugerem Uma probabilidade qualquer é definida como $\mathbb{P}(A)$ e definir ``P é uma probabilidade''. Mas neste material vou utilizar $P(:)$ por ser mais difundido em livros didáticos.}
	
\end{frame}
% ------------------------------------------------------------------------
\begin{frame}[t,fragile=singleslide]
\frametitle{Notação e regras básicas II, mais em \citep{Halperin1965}}
	\begin{columns}[T] % align columns
		\begin{column}{.85\textwidth}
			Costuma-se fazer referência ao espaço amostral como $\Omega$, assim $P(\Omega) = 1$			
			\vspace{0.5cm}
			Se um evento $A$ tem probabilidade $P(A)$ de ocorrer.\\
			$P(\bar{A}) = 1 - P(A)$ é a probabilidade do evento não ocorrer.\\
			\vspace{0.5cm}
			Dados dois eventos mutualmente independentes $A$ e $B$ (e.g., rodadas diferentes de um lançamento de moeda) e suas probabilidades $P(A)$ e $P(B)$, a probabilidade de ocorrerem:
			\begin{itemize}
				\item[-] $P(A)$ ou $P(B)$ é: $P(A \cup B) = P(A) + P(B)$ (um ou o outro ou os dois)
				\item[-] $P(A)$ e $P(B)$ é: $P(A \cap B) = P(A) \times P(B)$ (concomitantemente)
			\end{itemize} 
			\textbf{Independente}: Dois valores de uma mesma característica categórica, e.g., frequência fundamental grave ou aguda.\\
		\end{column}
		\hfill
		\begin{column}{0.05\textwidth}
		\end{column}		
	\end{columns}
	\vfill

	
\end{frame}
% ------------------------------------------------------------------------
\begin{frame}[t,fragile=singleslide]
\frametitle{Notação e regras básicas III, mais em \citep{Halperin1965}}
\begin{columns}[T] % align columns
	\begin{column}{.85\textwidth}
		Considere dois eventos \textbf{não} mutualmente independentes $A$ e $B$, como valores de características diferentes (e.g., frequência fundamental grave e presença de frênulo lingual) e suas  probabilidade $P(A)$ e $P(B)$. \\
		\vspace{0.5cm}
		A probabilidade  condicional, de ocorrer uma condição dada outra é:
		$P(B|A)$ lê-se $P(B)$ dado $A$\\
		\vspace{0.5cm}
		Neste caso:\\
		$P(A \cap B) = P(A) \times P(B|A)$.\\
		\vspace{0.5cm}
		Teorema de Bayes:\\
		\begin{equation*}
			P(B|A) = \frac{P(B)P(A|B)}{P(A)}
		\end{equation*}
	\end{column}
	\hfill
	\begin{column}{0.05\textwidth}
	\end{column}		
\end{columns}
\vfill

\end{frame}
% ------------------------------------------------------------------------
\begin{frame}[t,fragile=singleslide]
\frametitle{Probabilidades em Diagramas de Venn}
	\begin{tikzpicture}[remember picture,overlay]
	\node[xshift=0cm,yshift=0cm,opacity=1.0] at (current page.center) {\includegraphics[width=22cm]{venn-diagrams.png}};
	\end{tikzpicture}
	\vfill
	\lfr{\url{https://www.onlinemathlearning.com/image-files/venn-diagrams.png}}

\end{frame}
% ------------------------------------------------------------------------
\begin{frame}[t,fragile=singleslide]
\frametitle{Distribuições}
	Distribuição de uma variável discreta
	\begin{itemize}
		\item[-] $0 \leq P(x) \le 1$.
		\item[-] $\sum P(x) = 1$.
		\item[-] Função massa de probabilidade.
		\item[-] Probabilidade está diretamente em $P(x)$.
	\end{itemize} 
	\vspace{0.5cm}
	Distribuição de uma variável contínua
	\begin{itemize}
		\item[-] $0 \leq P(x) \le 1$. 
		\item[-] $\int\limits_{-\infty}^{\infty} p(x)dx = 1$.
		\item[-] Função densidade de probabilidade $p(x)$.
		\item[-] $P(1 \leq x \leq 3) = \int\limits_{1}^{3} p(x)dx$.
	\end{itemize} 
	\begin{tikzpicture}[remember picture,overlay]
	\node[xshift=8.5cm,yshift=4cm,opacity=1.0] at (current page.center) {\includegraphics[width=16cm]{prob-discreta-00.png}};
	\end{tikzpicture}
	\begin{tikzpicture}[remember picture,overlay]
	\node[xshift=8.5cm,yshift=-4cm,opacity=1.0] at (current page.center) {\includegraphics[width=16cm]{prob-continua-00.png}};
	\end{tikzpicture}

	\vfill
\end{frame}
% ------------------------------------------------------------------------
\begin{frame}[t,fragile=singleslide]
\frametitle{Parâmetros - vide \cite{Casella2011}}
	Valor esperado
	\begin{equation*}
		E[x] = \mu  = \sum\limits_{x \in X}xP(x) \quad \text{ou} \quad \mu = \int\limits_{x \in X}xp(x)dx
	\end{equation*}
	Variância
	\begin{equation*}
		E[(x-\mu)]^2 = \sigma^2 = \sum\limits_{x \in X}(x-\mu)^2P(x) \quad \text{ou} \quad  \sigma = \int\limits_{x \in X}(x-\mu)^2p(x)dx
	\end{equation*}
	Momentos estatísticos
	\begin{equation*}
	E[x]^n = \sum\limits_{x \in X}x^nP(x) \quad \text{ou} \quad  \int\limits_{x \in X}x^np(x)dx
	\end{equation*}
	Momentos centrais
	\begin{equation*}
	E[(x-\mu)]^n = \sum\limits_{x \in X}(x-\mu)^nP(x) \quad \text{ou} \quad  \int\limits_{x \in X}(x-\mu)^np(x)dx
	\end{equation*}
\end{frame}
% ------------------------------------------------------------------------
\begin{frame}[t,fragile=singleslide]
\frametitle{Função Normal}
	Padrão com $\mu = 0$ e $\sigma = 1$
	\begin{equation*}
		\mathcal{N}(x|0,1) = \frac{1}{\sqrt{2\pi}}\exp^{-\frac{x^2}{2}}
	\end{equation*}
	Geral, lembrar da regra do $\sigma$
	\begin{equation*}
		\mathcal{N}(x|\mu,\sigma) = \frac{1}{\sqrt{2\pi}\sigma}\exp^{-\frac{(x-\mu)^2}{2\sigma^2}}
	\end{equation*}
	\begin{tikzpicture}[remember picture,overlay]
	\node[xshift=0cm,yshift=-4cm,opacity=1.0] at (current page.center) {\includegraphics[width=20cm]{prob-normal-00.png}};
	\end{tikzpicture}
\end{frame}
% ------------------------------------------------------------------------
\begin{frame}[t,fragile=singleslide]
\frametitle{Probabilidade acumulada}
	\begin{tikzpicture}[remember picture,overlay]
	\node[xshift=0cm,yshift=3cm,opacity=1.0] at (current page.center) {\includegraphics[width=20cm]{prob-normal-acum-00.png}};
	\end{tikzpicture}
	
	\vspace{9.0cm}
	
	Covariância
	\begin{equation*}
	cov(x,y) = E\left[(x-\mu_x)(y-\mu_y) \right]
	\end{equation*}
	Correlação
	\begin{equation*}
	cov(x,y) = E\left[\frac{(x-\mu_x)}{\sigma_x}\frac{(y-\mu_y)}{\sigma_y} \right]
	\end{equation*}
\end{frame}
% ------------------------------------------------------------------------
\begin{frame}[t,fragile=singleslide]
\frametitle{Erro padrão}
	O erro padrão de uma \textbf{estatística} (na maioria das vezes a estimativa de um \textbf{parâmetro}) é o desvio padrão da distribuição amostral de uma estatística.
	
	Erro padrão da média:
	\begin{equation*}
		\sigma_{\bar{x}} = \frac{\sigma}{n}
	\end{equation*}
	
	Em geral, a distribuição amostral da média $\bar{x}$ tende a uma normal independente da distribuição de $X$.
	
	
	
\end{frame}
% ------------------------------------------------------------------------
\begin{frame}[t,fragile=singleslide]
\frametitle{Fim da Probabilidades - Dever de casa}
	\textbf{Exercícios do livro \cite{Agresti2018}:}
	\begin{itemize}
		\item[-] Capítulo 4: 4.1 - 4.7, 4.18 - 4.20, 4.26 - 4.32.
	\end{itemize}
\end{frame}
% ------------------------------------------------------------------------
\section[Estimação de Parâmetros]{Estimação de Parâmetros}
\begin{frame}[t,fragile=singleslide]
\frametitle{Definições}
	\begin{tikzpicture}[remember picture,overlay]
	\node[xshift=6cm,yshift=-4.5cm,opacity=1.0] at (current page.center) {\includegraphics[width=16cm]{hmfile_hash_f71959f8.jpg}};
	\end{tikzpicture}
		
\begin{itemize}
	\item[-] \textbf{Estimador pontual}: é uma melhor sugestão para um parâmetro.
	\item[-] \textbf{Estimador intervalar}: um intervalo ao redor da estimativa pontual em que acredita-se conter o parâmetro.
	\item[-] \textbf{Estimador não enviesado}: é centrado no parâmetro e possui a menor dispersão possível
	\item[*] Eficiente $\rightarrow$ se fechado ao redor do parâmetro.
\end{itemize}
	\lfr{\url{https://www.graphpad.com/guides/prism/latest/statistics/images/hmfile_hash_f71959f8.gif}}
\end{frame}

% ------------------------------------------------------------------------
\begin{frame}[t,fragile=singleslide]
\frametitle{Proporção}
	Uma proporção $\hat{p}$ calculada de N é uma estimativa não enviesada do parâmetro da proporção da população $\pi$. Na proporção:
	
	\begin{equation*}
		P(0) = 1-\hat{p}\qquad \text {e} \qquad P(1) = \hat{p}.
	\end{equation*}
	\begin{align*}
		\hat{m} &= 0 \cdot (1-\hat{p}) + 1 \cdot \hat{p} = \hat{p} \\
		\hat{s}^2 &= (0 - \hat{p})^2 \cdot (1-\hat{p}) + (1 - \hat{p})^2 \cdot \hat{p} = (1-\hat{p}) \cdot[\hat{p}\cdot(1 - \hat{p}) + \hat{p}^2] = \hat{p} \cdot (1 - \hat{p}) \\
		\hat{s}^2_{\mu} &= \frac{\hat{s}^2}{N} = {\frac{\hat{p} \cdot (1 - \hat{p})}{N}}\\
		N &= \hat{p} \cdot (1 - \hat{p}) \left( \frac{Z_\alpha}{M} \right)^2
	\end{align*}
	\begin{equation*}
		Z_{\frac{\alpha}{2}} \cdot \hat{s}{\mu} \leq \pi \leq Z_{1-\frac{\alpha}{2}} \cdot \hat{s}_{\mu}
	\end{equation*}

\end{frame}
% ------------------------------------------------------------------------
\begin{frame}[t,fragile=singleslide]
\frametitle{Z-score}
	\begin{tikzpicture}[remember picture,overlay]
	\node[xshift=0cm,yshift=0cm,opacity=1.0] at (current page.center) {\includegraphics[width=24cm]{normalstandard.jpg}};
	\end{tikzpicture}
	\lfr{\url{https://mathbitsnotebook.com/Algebra2/Statistics/normalstandard.jpg}}
\end{frame}
% ------------------------------------------------------------------------
\begin{frame}[t,fragile=singleslide]
\frametitle{Média}
	A média aritmética $\bar{m}$ é um estimador não enviesado da média populacional $\mu$ e pode ser calculada de N amostras:
	
	\begin{columns}[T] % align columns
		\begin{column}{.48\textwidth}
			\begin{align*}
			\bar{m} &= \frac{1}{N} \sum_{N} x_i \\
			\hat{s}^2 &= \frac{1}{N-1} \sum_{N} (x_i - \bar{m})^2 \\
			\hat{s}^2_{\mu} &= \frac{\hat{s}^2}{N} \\
			N &= \hat{s}^2 \left( \frac{t_{(\frac{\alpha}{2},N-1)}}{\delta^*} \right)^2
			\end{align*}
			\begin{equation*}
			t_{(\frac{\alpha}{2},N-1)} \cdot \hat{s}_{\mu} \leq \mu \leq t_{(1-\frac{\alpha}{2},N-1)} \cdot \hat{s}_{\mu}
			\end{equation*}
		\end{column}		
		\begin{column}{.48\textwidth}
			$\alpha$:  proporção da FDP ou significância. \\
			
			A confiança $\gamma = 1 - \alpha$. \\
			
			$M$ é proporção de erro aceitável (e.g. se erro 3\% M = 0.003).\\
			
			$\delta^*$ é o mínimo efeito de interesse, na dimensão da variável (e.g. 5 gramas em medidas de massa).
		\end{column}
	\end{columns}
	
\end{frame}
% ------------------------------------------------------------------------
\begin{frame}[t,fragile=singleslide]
\frametitle{Distribuição t-Student}
	\begin{tikzpicture}[remember picture,overlay]
	\node[xshift=0cm,yshift=0cm,opacity=1.0] at (current page.center) {\includegraphics[width=24cm]{t-vs-z_orig.png}};
	\end{tikzpicture}
	\lfr{\url{https://mri-q.com/uploads/3/4/5/7/34572113/t-vs-z_orig.png}}
\end{frame}
% ------------------------------------------------------------------------
\begin{frame}[t,fragile=singleslide]
\frametitle{Desvio padrão e testes }
	O intervalo de estimativa do desvio padrão 
	
	\begin{equation*}
	\frac{N-1}{\chi^2_{(\frac{\alpha}{2},N-1)}} \cdot \hat{s}^2 \leq \sigma^2 \leq \frac{N-1}{ \chi^2_{(1- \frac{\alpha}{2},N-1)}} \cdot \hat{s}^2
	\end{equation*}
	
	Onde $\chi^2_{(\alpha,N-1)}$ é a distribuição qui-quadrado na proporção $\alpha$ com N-1 graus de liberdade.
	\vspace{2cm}	
	Métodos de estimativa baseado em subamostragens
	\begin{itemize}
		\item[-] \textit{bootstrap}
		\item[-] \textit{jack knife}
	\end{itemize}
\end{frame}
% ------------------------------------------------------------------------
\begin{frame}[t,fragile=singleslide]
\frametitle{Fim de Estimação de parâmetros - Dever de casa}
\textbf{Exercícios do livro \cite{Agresti2018}:}
\begin{itemize}
	\item[-] Capítulo 5: 5.1, 5.9, 5.24, 5.25, 5.33, 5.34, 5.36, 5.40.
\end{itemize}
\end{frame}
% ------------------------------------------------------------------------
\section[Teste de Significância]{Teste de Significância}
\begin{frame}[t,fragile=singleslide]
\frametitle{Premissas e definições}
	\begin{itemize}
		\item[-] \textbf{hipótese}: Uma declaração sobre uma população (e.g. falantes de Minas Gerais utilizam frequentemente o termo ``trem'' como palavra ônibus.).
		\item[-] \textbf{hipótese estatística}: declaração sobre um parâmetro da população (e.g. a \textbf{média} do uso do termo ``trem'' como palavra ônibus por falantes de Minas Gerais é superior a média das demais).
		\item[-] \textbf{Teste de significância}: utiliza os dados para construir/sintetizar uma evidencia sobre uma hipótese.
		\item[-] \textbf{Premissas}:
		\begin{itemize}
			\item[*] dados qualitativos ou categóricos;
			\item[*] aleatorização;
			\item[*] amostra segue a distribuição da população;
			\item[*] tamanho da amostra.
		\end{itemize}
	\end{itemize}

\end{frame}
% ------------------------------------------------------------------------
\begin{frame}[t,fragile=singleslide]
\frametitle{Premissas e definições}

	\textbf{Hipótese nula $H_0$}: Declaração sobre um parâmetro da população para um \textbf{valor em particular} (neste caso hipótese precisa.). Representa um estado de não influência ou ausência de efeito. 
	
	\textbf{Hipótese nula $H_1$ ou $H_\alpha$}: Representa o efeito. 
	
	Exemplo: Consideremos um estudo em que diferentes falantes de Minas Gerais são expostos a diferentes cenas para depois descrevê-las. Na descrição contabiliza-se o uso de palavras ônibus dividindo-as em 5 categorias:
	
	\begin{itemize}
		\item[-] trem (1);
		\item[-] troço (2);
		\item[-] coisa (3);
		\item[-] negócio (4);
		\item[-] outras (5).
	\end{itemize}	
	
\end{frame}
% ------------------------------------------------------------------------
\begin{frame}[t,fragile=singleslide]
\frametitle{Escrevendo hipóteses}

\textbf{hipótese nula:} não existe diferença estatística entre a ocorrência da palavra ``trem'' frente as demais categorias, ou $\mu_1 = \mu_2 = \mu_3 = \mu_4 = \mu_5$; \\

\textbf{hipótese alternativa:} ocorrência média da palavra ``trem'' é maior que a média da ocorrência de cada uma das demais, ou $\mu_1 > \mu_2 \wedge \mu_1 > \mu_3 \wedge \mu_1 > \mu_4 \wedge \mu_1 > \mu_5$.
\vspace{1cm}
\begin{equation*}
\left\{ \begin{array}{cl}
H_0: & \mu_1 = \mu_2 = \mu_3 = \mu_4 = \mu_5 \\
H_1: & \mu_1 > \mu_2 \wedge \mu_1 > \mu_3 \wedge \mu_1 > \mu_4 \wedge \mu_1 > \mu_5
\end{array}\right.
\end{equation*}
\linebreak
\ul{Observação 1}: Esta é uma comparação entre cinco categorias sendo que o método para realizar esta comparação e testar as hipóteses é um pouco mais elaborado.

\ul{Observação 2}: as mesmas hipóteses poderiam ser escritas na forma de proporção.
\end{frame}
% ------------------------------------------------------------------------
\begin{frame}[t,fragile=singleslide]
\frametitle{Compreender para argumentar}

O teste verifica quão provável é $H_0$, desta forma ele permite \ul{rejeitar} $H_0$ com um nível de significância $\alpha$ ou \ul{falha em rejeitar} $H_0$. Princípio do \textit{onus probandi}.

\vspace{1.5cm}
\begin{table}[]
	\centering
	\begin{tabular}{|cc|cc|}
		\hline
		\multicolumn{2}{|c|}{\multirow{2}{*}{\begin{tabular}[c]{@{}c@{}}$H_0$/\\ não $H_1$ (efeito)\end{tabular}}}                                                                                    & \multicolumn{2}{c|}{Condição ou Realidade}                                                                                                                       \\ \cline{3-4} 
		\multicolumn{2}{|c|}{}                                                                                                                                                                        & \multicolumn{1}{c|}{\begin{tabular}[c]{@{}c@{}}Verdadeira/\\ Efeito Negativo\end{tabular}} & \begin{tabular}[c]{@{}c@{}}Falsa/\\ Efeito Positivo\end{tabular}    \\ \hline
		\multicolumn{1}{|c|}{\multirow{2}{*}{\begin{tabular}[c]{@{}c@{}}Decisão/ Resultado\\ do teste\end{tabular}}} & \begin{tabular}[c]{@{}c@{}}Verdadeira/\\ Efeito Negativo\end{tabular} & \multicolumn{1}{c|}{\begin{tabular}[c]{@{}c@{}}Verdadeiro\\ negativo (TN)\end{tabular}}    & \begin{tabular}[c]{@{}c@{}}Falso \\ Negativo (FN)\end{tabular}      \\ \cline{2-4} 
		\multicolumn{1}{|c|}{}                                                                                                & \begin{tabular}[c]{@{}c@{}}Falsa/\\ Efeito Positivo\end{tabular}      & \multicolumn{1}{c|}{\begin{tabular}[c]{@{}c@{}}Falso\\ Positivo (FP)\end{tabular}}         & \begin{tabular}[c]{@{}c@{}}Verdadeiro \\ positivo (TP)\end{tabular} \\ \hline
	\end{tabular}
\end{table}
\end{frame}
% ------------------------------------------------------------------------
\begin{frame}[t,fragile=singleslide]
\frametitle{Decisões}
	Na realização do teste existem algumas diferenças metodológicas entre o que foi proposto por Fisher e por Neyman–Pearson. 
	Basicamente:
	\begin{itemize}
		\item[-]  Escolhe-se o limiar de decisão, ou significância $\alpha$. A confiabilidade do teste é $\gamma = 1 - \alpha$.
		\item[-] Calcula-se o valor-p, que é a probabilidade de obter repetibilidade do resultado (ou mais extremo) sob a condição da hipótese nula ser correta (mnemônico: credibilidade de $H_0$).
		\item[-] Se o valor-p for menor que a significância, rejeita-se $H_0$ com significância $\alpha$ (ou de confiança $1-\alpha$).
	\end{itemize} 
\end{frame}	
% ------------------------------------------------------------------------
\begin{frame}[t,fragile=singleslide]
\frametitle{Erro tipo I e Erro tipo II}
Erros:
\begin{itemize}
	\item[-] Erro do Tipo I (falso positivo), rejeitar $H_0$ quando ela é verdadeira. $P(\text{rejeitar }H_0|H_0=V) = \alpha$
	\item[-] Erro do Tipo II (falso negativo), falhar em rejeitar $H_0$ quando ela é falsa. $P(\text{falha em rejeitar }H_0|H_1=V) = \beta$
\end{itemize} 
\vspace{1.cm}
\begin{table}[]
	\centering
	\begin{tabular}{|cc|cc|}
		\hline
		\multicolumn{2}{|c|}{\multirow{2}{*}{\begin{tabular}[c]{@{}c@{}}$H_0$/\\ não $H_1$ (efeito)\end{tabular}}}                                                                             & \multicolumn{2}{c|}{Condição ou Realidade}                                                                                                                                      \\ \cline{3-4} 
		\multicolumn{2}{|c|}{}                                                                                                                                                                 & \multicolumn{1}{c|}{\begin{tabular}[c]{@{}c@{}}Verdadeira/\\ Efeito Negativo\end{tabular}}       & \begin{tabular}[c]{@{}c@{}}Falsa/\\ Efeito Positivo\end{tabular}             \\ \hline
		\multicolumn{1}{|c|}{\multirow{2}{*}{\begin{tabular}[c]{@{}c@{}}Decisão/\\ Resultado\\ do teste\end{tabular}}} & \begin{tabular}[c]{@{}c@{}}Verdadeira/\\ Efeito Negativo\end{tabular} & \multicolumn{1}{c|}{\begin{tabular}[c]{@{}c@{}}TN\\ ($1-\alpha$)\end{tabular}}      & \begin{tabular}[c]{@{}c@{}}Erro tipo II \\ $\beta$\end{tabular} \\ \cline{2-4} 
		\multicolumn{1}{|c|}{}                                                                                         & \begin{tabular}[c]{@{}c@{}}Falsa/\\ Efeito Positivo\end{tabular}      & \multicolumn{1}{c|}{\begin{tabular}[c]{@{}c@{}}Erro tipo I\\ $\alpha$\end{tabular}} & \begin{tabular}[c]{@{}c@{}}TP\\ ($1-\beta$)\end{tabular}          \\ \hline
	\end{tabular}
\end{table}
\end{frame}	
% ------------------------------------------------------------------------
\begin{frame}[t,fragile=singleslide]
\frametitle{Poder do teste}
\ul{Poder do teste} ($1-\beta$): Capacidade/probabilidade de rejeitar $H_0$ quando uma hipótese alternativa específica $H_1$ é verdadeira. 
	\begin{tikzpicture}[remember picture,overlay]
	\node[xshift=0cm,yshift=-2cm,opacity=1.0] at (current page.center) {\includegraphics[width=22cm]{type-I-error.jpg}};
	\end{tikzpicture}
	\lfr{\url{https://images.deepai.org/glossary-terms/e5da790fe351430592c2332f693dc8e6/type-I-error.jpg}}
\end{frame}
% ------------------------------------------------------------------------
\begin{frame}[t,fragile=singleslide]
\frametitle{Teste de uma proporção com R - Bilateral}
	Dada uma amostra de dois grupos de pessoas, um com disfluência na fala e outro sem. Vamos colocar algumas hipóteses...
	
	\textbf{Hipótese de calda dupla}
	A proporção observada de pessoas do sexo masculino é diferente de 0,5?
	\begin{equation*}
	\left\{ \begin{array}{cl}
	H_0: & \pi = 0,5 \\
	H_1: & \pi \neq 0,5
	\end{array}\right.
	\end{equation*}
	
	\begin{lstlisting}
	prop.test(x=18, n=40, p=0.50, alternative="two.sided")
	
	data:  18 out of 40, null probability 0.5
	X-squared = 0.225, df = 1, p-value = 0.6353
	alternative hypothesis: true p is not equal to 0.5
	95 percent confidence interval:
	0.2960304 0.6134103
	sample estimates:
	p 
	0.45
	\end{lstlisting}
	
	\lfr{O exemplo utiliza dos dados da pesquisa não publicada (por enquanto) da aluna Ana Vieira e Beatriz de Matos.}
\end{frame}
% ------------------------------------------------------------------------
\begin{frame}[t,fragile=singleslide]
\frametitle{Teste de uma proporção com R - Unilateral superior}
	\textbf{Hipótese de superioridade}
	A proporção observada de pessoas do sexo masculino é maior a 0,5?
	\begin{equation*}
	\left\{ \begin{array}{cl}
	H_0: & \pi = 0,5 \\
	H_1: & \pi > 0,5
	\end{array}\right.
	\end{equation*}
	
	\begin{lstlisting}
	prop.test(x=18, n=40, p=0.50, alternative="greater")
	
	data:  18 out of 40, null probability 0.5
	X-squared = 0.225, df = 1, p-value = 0.6824
	alternative hypothesis: true p is greater than 0.5
	95 percent confidence interval:
	0.3165333 1.0000000
	sample estimates:
	p 
	0.45 
	\end{lstlisting}

\lfr{O exemplo utiliza dos dados da pesquisa não publicada (por enquanto) da aluna Ana Vieira e Beatriz de Matos.}
\end{frame}
% ------------------------------------------------------------------------
\begin{frame}[t,fragile=singleslide]
\frametitle{Teste de uma proporção com R - Unilateral inferior}
	\textbf{Hipótese de inferioridade}
	A proporção observada de pessoas do sexo masculino é menor a 0,5?
	\begin{equation*}
	\left\{ \begin{array}{cl}
	H_0: & \pi = 0,5 \\
	H_1: & \pi < 0,5
	\end{array}\right.
	\end{equation*}
	
	\begin{lstlisting}
	prop.test(x=18, n=40, p=0.50, alternative="less")
	
	data:  18 out of 40, null probability 0.5
	X-squared = 0.225, df = 1, p-value = 0.3176
	alternative hypothesis: true p is less than 0.5
	95 percent confidence interval:
	0.0000000 0.5903943
	sample estimates:
	p 
	0.45 
	\end{lstlisting}

\lfr{O exemplo utiliza dos dados da pesquisa não publicada (por enquanto) da aluna Ana Vieira e Beatriz de Matos.}
\end{frame}
% ------------------------------------------------------------------------
\begin{frame}[t,fragile=singleslide]
\frametitle{Teste sobre a média com R - Bilateral}
	Dada uma amostra de dois grupos de pessoas, um com disfluência na fala e outro sem. Vamos colocar algumas hipóteses...
	
	\textbf{Hipótese de calda dupla}
	A idade média dos participantes é diferente a 35 anos.
	\begin{equation*}
	\left\{ \begin{array}{cl}
	H_0: & \hat{\mu} = 35 \\
	H_1: & \hat{\mu} \neq 35
	\end{array}\right.
	\end{equation*}
	
	\begin{lstlisting}
	t.test(x=df_disf$IDADE, mu=35,alternative='two.sided')
	
	data:  df_disf$IDADE
	t = -4.0621, df = 39, p-value = 0.0002274
	alternative hypothesis: true mean is not equal to 35
	95 percent confidence interval:
	26.31193 32.08807
	sample estimates:
	mean of x 
	29.2  
	\end{lstlisting}
	
	\lfr{O exemplo utiliza dos dados da pesquisa não publicada (por enquanto) da aluna Ana Vieira e Beatriz de Matos.}

\end{frame}
% ------------------------------------------------------------------------
\begin{frame}[t,fragile=singleslide]
\frametitle{Teste sobre a média com R - Unilateral superior}
\textbf{Hipótese de superioridade}
A idade média dos participantes é superior a 35 anos.
\begin{equation*}
\left\{ \begin{array}{cl}
H_0: & \hat{\mu} = 35 \\
H_1: & \hat{\mu} > 35
\end{array}\right.
\end{equation*}

\begin{lstlisting}
t.test(x=df_disf$IDADE, mu=35,alternative='greater')

t = -4.0621, df = 39, p-value = 0.9999
alternative hypothesis: true mean is greater than 35
95 percent confidence interval:
26.79427      Inf
sample estimates:
mean of x 
29.2 
\end{lstlisting}

\lfr{O exemplo utiliza dos dados da pesquisa não publicada (por enquanto) da aluna Ana Vieira e Beatriz de Matos.}

\end{frame}
% ------------------------------------------------------------------------
\begin{frame}[t,fragile=singleslide]
\frametitle{Teste sobre a média com R  - Unilateral inferior}
\textbf{Hipótese de superioridade}
A idade média dos participantes é inferior a 35 anos.
\begin{equation*}
\left\{ \begin{array}{cl}
H_0: & \hat{\mu} = 35 \\
H_1: & \hat{\mu} < 35
\end{array}\right.
\end{equation*}

\begin{lstlisting}
t.test(x=df_disf$IDADE, mu=35,alternative='less')

data:  df_disf$IDADE
t = -4.0621, df = 39, p-value = 0.0001137
alternative hypothesis: true mean is less than 35
95 percent confidence interval:
-Inf 31.60573
sample estimates:
mean of x 
29.2 
\end{lstlisting}

\lfr{O exemplo utiliza dos dados da pesquisa não publicada (por enquanto) da aluna Ana Vieira e Beatriz de Matos.}

\end{frame}
% ------------------------------------------------------------------------
\begin{frame}[t,fragile=singleslide]
\frametitle{Poder do teste e número de amostras}

\begin{lstlisting}
library(pwr)
sdIdade <- sd(df_disf$IDADE)
delta <- 3/sdIdade  # Minimo efeito (detectavel). Exemplo 3 anos na media de idade
alpha <- 0.05 # Significancia
power <- 0.80 # 1-beta

pwr.t.test(d = delta, sig.level = alpha, power = power, type = "one.sample", alternative = "two.sided")

One-sample t test power calculation 

n = 73.06228
d = 0.33221
sig.level = 0.05
power = 0.8
alternative = two.sided
\end{lstlisting}

\lfr{O exemplo utiliza dos dados da pesquisa não publicada (por enquanto) da aluna Ana Vieira e Beatriz de Matos.}

\end{frame}
% ------------------------------------------------------------------------
\begin{frame}[t,fragile=singleslide]
\frametitle{Fim de Teste de Significância - Dever de casa}
\textbf{Exercícios do livro \cite{Agresti2018}:}
\begin{itemize}
	\item[-] Capítulo 6: 6.1-6.5, 6.17, 6.23, 6.41.
\end{itemize}
\end{frame}
% ------------------------------------------------------------------------
\section{Comparação de dois grupos}
\begin{frame}[t,fragile=singleslide]
\frametitle{Busca por uma diferença ou razão entre parâmetros}
\vspace{0.5cm}
\begin{itemize}
	\item[-] Amostras independentes ou dependentes (pareada).
	\item[-] Variáveis explicatórias vs. variável resposta.
	\item[-] Conhecer ou não alguns parâmetros da população.
\end{itemize}
\vspace{0.5cm}
Exemplo: Dada uma amostra de dois grupos de pessoas, um com disfluência na fala e outro sem. 
\textbf{Hipótese:} A média do número de gestos que cada grupo realiza é igual?

Sendo:\\
$\mu_1$ é a média de gestos do grupo com disfluência; e \\
$\mu_2$ a média de gestos do grupo controle. \\

\begin{equation*}
\left\{ \begin{array}{cl}
H_0: & \mu_2 - \mu_1 = 0 \\
H_1: & \mu_2 - \mu_1 \neq 0
\end{array}\right.
\end{equation*}

\lfr{O exemplo utiliza dos dados da pesquisa não publicada (por enquanto) da aluna Ana Vieira e Beatriz de Matos.}
\end{frame}
% ------------------------------------------------------------------------
\begin{frame}[t,fragile=singleslide]
\frametitle{Amostras independentes ou dependentes}
	\textbf{Independente:} 	Cada unidade experimental (voluntário) passou por um tratamento diferente ou está em um grupo diferente.\\
	\textbf{Dependente:} A unidade experimental (gravação de voz, foto, voluntário) pode ser submetido a mais de um tratamento.
	\vspace{0.5cm}
	No exemplo
	\begin{table}[]
		\centering
		\begin{tabular}{lccc}
			&         & \multicolumn{2}{c}{Número de gestos} \\ \cline{3-4} 
			Grupo     & Amostras & Média         & Desvio padrão        \\ \hline
			Disfunção & 20       & 5,8           & 7,9                  \\
			Controle  & 20       & 12,7          & 9,4                 
		\end{tabular}
	\end{table}


\end{frame}
% ------------------------------------------------------------------------
\begin{frame}[t,fragile=singleslide]
\frametitle{Calculando o teste para duas amostras}

	Passos: 
	\begin{itemize}
		\item[-] verificar se as variâncias são iguais com o \textit{var.test()};
		\item[-] indicar se as variâncias dos dois grupos são iguais (e.g. \textit{var.equal = (varTest\$p.value >= 0.05)});
		\item[-] selecionar o nível de confiança \textit{conf.level = 0.95};
		\item[-] indicar se o teste é pareado (e.g. \textit{paired = FALSE});
		\item[-] a hipótese da média (e.g. \textit{mu = 0});
	\end{itemize}

	\begin{lstlisting}
	library(pwr)
	library(effsize)
	
	varTest <- var.test(df_disf$Ges_Total ~df_disf$GRUPO, alternative = "two.sided")
	t.test(df_disf$Ges_Total ~df_disf$GRUPO, mu = 0, paired = FALSE, 
				conf.level = 0.95, var.equal = (varTest$p.value >= 0.05))
	cohenD <- cohen.d(df_disf$Ges_Total,df_disf$GRUPO)
	pwr.t.test(n=20, d=cohenD$estimate, type = "two.sample", alternative = "two.sided")
	\end{lstlisting}
\end{frame}
% ------------------------------------------------------------------------
\begin{frame}[t,fragile=singleslide]
\frametitle{Resultado}
	
	Médias são diferentes com p-valor = 0,01669
	
	\vspace{1cm}
	
	\begin{lstlisting}
		Two Sample t-test
	
	data:  df_disf$Ges_Total by df_disf$GRUPO
	t = -2.504, df = 38, p-value = 0.01669
	alternative hypothesis: true difference in means is not equal to 0
	95 percent confidence interval:
	-12.478357  -1.321643
	sample estimates:
	mean in group 1 mean in group 2 
	5.8            12.7 
	\end{lstlisting}
	
	
\end{frame}
% ------------------------------------------------------------------------
\begin{frame}[t,fragile=singleslide]
\frametitle{Potência do teste}

	Sendo a potência do teste é de 0,9527, ou seja, a probabilidade de cometer o erro do tipo II é de 4,73\%.
	
	\vspace{1cm}
	
	\begin{lstlisting}
	Two-sample t test power calculation 
	
	n = 20
	d = 1.178793
	sig.level = 0.05
	power = 0.9527323
	alternative = two.sided
	
	NOTE: n is number in *each* group
	
	\end{lstlisting}
\end{frame}
% ------------------------------------------------------------------------
\begin{frame}[t,fragile=singleslide]
\frametitle{Cohen-d}
	
	Diferença média padronizada. É uma medida do tamanho do efeito, ou a distância (no exemplo em número de gestos) entre os grupos:
	\vspace{0.5cm}
	\begin{lstlisting}
	Cohen's d
	d estimate: 1.178793 (large)
	95 percent confidence interval:
	lower     upper 
	0.6965114 1.6610739 
	\end{lstlisting}
	
	\begin{flalign*}
	&d = \frac{\bar{x}_1 - \bar{x}_2}{\sqrt{\frac{(N_1-1)s^2_1 + (N_2-1)s^2_2}{N_1 + N_2 - 2}}} & \\
	\end{flalign*}
	
	\begin{tikzpicture}[remember picture,overlay]
	\node[xshift=8cm,yshift=-2cm,opacity=1.0] at (current page.center) {\includegraphics[width=12cm]{cohens-d-hypothesized-mean-difference.png}};
	\end{tikzpicture}
	\lfr{\url{https://www.statisticshowto.com/cohens-d/}}
\end{frame}
% ------------------------------------------------------------------------
\begin{frame}[t,fragile=singleslide]
	\frametitle{Calculando o teste para duas proporções}

	Exemplo: Dada uma amostra de dois grupos de pessoas, um com disfluência na fala e outro sem. 
	\textbf{Hipótese:} Considerando que a velocidade de fala é dividida em duas categorias ``Adequado e Alterado'', a proporção de fala adequada é maior no grupo controle?
	
	\begin{table}[]
		\centering
		\begin{tabular}{lccc}
			& \multicolumn{2}{c}{Velocidade de fala} &       \\ \cline{2-3}
			Grupo     & Adequado           & Alterado          & Total \\ \hline
			Disfunção & 13 (0,65)          & 7 (0,35)          & 20    \\
			Controle  & 19 (0,95)          & 1 (0,05)          & 20   
		\end{tabular}
	\end{table}

	Sendo:\\
	$\pi_1$ é a proporção de falas adequadas no grupo com disfluência; e \\
	$\pi_2$ a proporção de falas adequadas no grupo controle. \\
	
	\begin{equation*}
	\left\{ \begin{array}{cl}
	H_0: & \pi_2 - \pi_1 = 0 \\
	H_1: & \pi_2 - \pi_1 \neq 0
	\end{array}\right.
	\end{equation*}

	\lfr{O exemplo utiliza dos dados da pesquisa não publicada (por enquanto) da aluna Ana Vieira e Beatriz de Matos.}

\end{frame}	
% ------------------------------------------------------------------------
\begin{frame}[t,fragile=singleslide]
\frametitle{Realizando o teste}
	Passos: 
	\begin{itemize}
		\item[-] separar o numero de eventos de sucesso (e.g. fala ``Adequada'') no número de amostras \textit{x=c(13,19)} e \textit{n=c(20,20)};
		\item[-] selecionar o nível de confiança \textit{conf.level = 0.95};
		\item[-] correção de continuidade de Yates's para poucas amostras (e.g. \textit{correct = TRUE});
	\end{itemize}
	
	\begin{lstlisting}
	library(pwr)
	
	propTest <- prop.test(x=c(13,19), n=c(20,20),p = NULL, alternative = "two.sided",
	correct = TRUE)
	pwr.2p.test(h = sqrt(propTest$statistic/20), n = 20, sig.level = 0.05, power = NULL, 
	alternative = "two.sided")
	\end{lstlisting}

\end{frame}


% ------------------------------------------------------------------------
\begin{frame}[t,fragile=singleslide]
	\frametitle{Potência do teste e tamanho efeito}
	
	Sendo a potência do teste é de 0,287, ou seja, a probabilidade de cometer o erro do tipo II é de 71,3\%.
	
	\begin{flalign*}
	&h = \sqrt{\frac{\chi^2}{n}} & \\
	\end{flalign*}
	
	\begin{lstlisting}
	Difference of proportion power calculation for binomial distribution (arcsine transformation) 
	
	h = 0.4419417
	n = 20
	sig.level = 0.05
	power = 0.2873077
	alternative = two.sided
	
	NOTE: same sample sizes
	\end{lstlisting}
	\end{frame}
% ------------------------------------------------------------------------
\begin{frame}[t,fragile=singleslide]
\frametitle{Fim de Comparação de dois grupos - Dever de casa}
\textbf{Exercícios do livro \cite{Agresti2018}:}
\begin{itemize}
	\item[-] Capítulo 7: 7.4, 7.11, 7.14, 7.21, 7.28, 7.47.
\end{itemize}
\end{frame}
% ------------------------------------------------------------------------
% ------------------------------------------------------------------------
\section{Associação de Variáveis Categóricas}
\begin{frame}[t,fragile=singleslide]
\frametitle{Tabela de contingência}

	Uma empresa de desenvolvimento de cremes faciais realiza um experimento com 600 pessoas. Trezentas (300) no ``grupo de controle'' e 300 no ``grupo de tratamento'' \citep{Kahan2017}.	
	
	\begin{table}[]
		\begin{tabular}{C{6cm}|C{4cm}|C{4cm}|C{4cm}|C{4cm}}
			& Melhora & sem efeito & Piora & Total \\ \midrule
			Controle   & 107     & 170        & 23    & 300   \\ \midrule
			Tratamento & 223     & 2          & 75    & 300   \\ \midrule
			Total      & 330     & 172        & 98    & 600     
		\end{tabular}
	\end{table}



\end{frame}
% ------------------------------------------------------------------------
\begin{frame}[t,fragile=singleslide]
\frametitle{Variáveis são independentes}
\vspace{1cm}
Uso do teste qui-quadrado, sendo: \\

\begin{equation*}
\left\{ \begin{array}{cl}
H_0 : \text{As variáveis são estatisticamente independentes}.\\
H_1 : \text{As variáveis são estatisticamente dependentes}.
\end{array}\right.
\end{equation*}

\vspace{1cm}
Quando não utilizar o $\chi^2$:
\begin{itemize}
	\item[-] Quando existir \textbf{dependência} entre as \ul{observações}.
	\item[-] Em caso de células (posições da tabela de contingência) com menos de 5 observações.
	\item[-] Variáveis não categóricas.
\end{itemize}
\vspace{1cm}
\textbf{Obs}: o teste também pode ser utilizado para ajuste de modelos ou homogeneidade.

\end{frame}
% ------------------------------------------------------------------------
\begin{frame}[t,fragile=singleslide]
\frametitle{Teste $\chi^2$}

	\begin{lstlisting}
	tabABC <- as.table(rbind(c(223,2,75), c(107,170,23)))
	dimnames(tabABC) <- list(grupos = c("Tratamento", "Controle"), resultado = c("Melhora","Sem efeito", "Piora"))
	
	            resultado
	grupos       Melhora Sem efeito Piora
	Tratamento     223          2    75
	Controle       107        170    23
	
	            resultado
	grupos       Melhora Outro
	Tratamento     223    77
	Controle       107   193
	
	            resultado
	grupos       Outro Piora
	Tratamento   225    75
	Controle     277    23
	\end{lstlisting}
\end{frame}
% ------------------------------------------------------------------------
\begin{frame}[t,fragile=singleslide]
\frametitle{Medindo o teste e o tamanho do efeito}
	Calculando os testes para cada tabela:
	\vspace{0.5cm}
	\begin{lstlisting}
		chsqtabAB <- chisq.test(tabAB)
		chsqtabBC <- chisq.test(tabBC)
		contribAB <- 100*chsqtabAB$residuals^2/chsqtabAB$statistic
		contribBC <- 100*chsqtabBC$residuals^2/chsqtabBC$statistic
	\end{lstlisting}
	\vspace{0.5cm}
	\textbf{Resíduos:} Indicam as variáveis que apresentam maior poder de explicação. Sinal indica a direção de (a favor ou contra o tratamento).\\
	\vspace{0.5cm}
	Tamanho do efeito e contribuição:
	\begin{flalign*}
	&\Psi = \sqrt{\frac{\chi^2}{n}} &  c = \frac{r^2}{\chi^2} & & \\ 
	\end{flalign*}
\end{frame}
% ------------------------------------------------------------------------
\begin{frame}[t,fragile=singleslide]
\frametitle{Resíduos e tamanho do efeito}
	\begin{columns}[T] % align columns
		\begin{column}{.48\textwidth}
			\begin{lstlisting}
			            resultado
			grupos       Melhora Outro
			Tratamento     223    77
			Controle       107   193
			
			Pearson's Chi-squared test with Yates' continuity correction
			
			data:  tabAB
			X-squared = 89.057, df = 1, p-value < 2.2e-16
			
			            resultado
			grupos       Melhora Outro
			Tratamento   22.89 	27.98
			Controle     22.89 	27.98
			\end{lstlisting}
		\end{column}%
		\hfill%
		\begin{column}{.48\textwidth}	
			\begin{lstlisting}
			            resultado
			grupos       Outro Piora
			Tratamento   225    75
			Controle     277    23
			
			Pearson's Chi-squared test with Yates' continuity correction
			
			data:  tabBC
			X-squared = 31.722, df = 1, p-value = 1.779e-08
			
			            resultado
			grupos       Outro Piora
			Tratamento  8.49 	43.49
			Controle    8.49 	43.49
			\end{lstlisting}			
		\end{column}%
	\end{columns}

\end{frame}
% ------------------------------------------------------------------------
\begin{frame}[t,fragile=singleslide]
\frametitle{Fim de Associação de Variáveis Categóricas - Dever de casa}
\textbf{Exercícios do livro \cite{Agresti2018}:}
\begin{itemize}
	\item[-] Capítulo 8: 8.5, 8.16, 8.22, 8.40.
\end{itemize}
\end{frame}
% ------------------------------------------------------------------------
\section{Regressão Linear e Correlação}
\begin{frame}[t,fragile=singleslide]
\frametitle{Fim de Regressão Linear e Correlação}
\vspace{0.5cm}
\begin{itemize}
	\item[-] Associação entre variáveis quantitativas.
	\item[-] Relação linear ou linearizáveis.
	\item[-] Importância da visualização de dados
\end{itemize}

	\begin{tikzpicture}[remember picture,overlay]
		\node[xshift=4cm,yshift=-2.5cm,opacity=1.0] at (current page.center) {\includegraphics[width=14cm]{featured.jpg}};
	\end{tikzpicture}

	\lfr{\url{https://www.tiagoms.com/project/regressaolinear/featured.jpg}}
\end{frame}
% ------------------------------------------------------------------------
\begin{frame}[t,fragile=singleslide]
\frametitle{Dados do Corpus Britânico \cite{Levshina2015}}

\begin{lstlisting}
reg_bnc <- read.table("./reg_bnc.csv", header = TRUE, sep = ",",dec = ".",quote = "\"")
png(file = "./Correlacao_01.png",width = 600, height = 450, units = "px")
	ggcorrplot(cor(reg_bnc[3:12]))
dev.off()

Reg
Acad      : 6   
Fiction   : 3   
Misc      :14   
News      :16   
NonacProse: 6   
Spok      :24 
\end{lstlisting}
	\begin{tikzpicture}[remember picture,overlay]
	\node[xshift=4cm,yshift=-2.5cm,opacity=1.0] at (current page.center) {\includegraphics[width=16cm]{../Rcodes/Correlacao_01.png}};
	\end{tikzpicture}

\end{frame}

% ------------------------------------------------------------------------
\begin{frame}[t,fragile=singleslide]
\frametitle{Observando a dispersão e a normalidade}
	\begin{tikzpicture}[remember picture,overlay]
	\node[xshift=0cm,yshift=-1cm,opacity=1.0] at (current page.center) {\includegraphics[width=22cm]{../Rcodes/Multiplot_01.png}};
	\end{tikzpicture}
\end{frame}
% ------------------------------------------------------------------------
\begin{frame}[t,fragile=singleslide]
\frametitle{Análise da correlação - positiva}
	\vspace{0.5cm}
	Teste de correlação: Ncomm vs Adj (positiva)
	\vspace{0.5cm}
	\begin{tikzpicture}[remember picture,overlay]
	\node[xshift=8cm,yshift=0cm,opacity=1.0] at (current page.center) {\includegraphics[width=14cm]{../Rcodes/Scater_01.png}};
	\end{tikzpicture}
	
\begin{lstlisting}
Pearson's product-moment correlation

data:  reg_bnc$Ncomm and reg_bnc$Adj
t = 14.009, df = 67, p-value < 2.2e-16
alternative hypothesis: true correlation is not equal to 0
95 percent confidence interval:
0.7877011 0.9134309
sample estimates:
cor 
0.8634126 
\end{lstlisting}
\end{frame}
% ------------------------------------------------------------------------
\begin{frame}[t,fragile=singleslide]
\frametitle{Análise da regressão - inclinação positiva}
	\vspace{1cm}
	Analisando o resultado da regressão
	\vspace{0.5cm}
	\begin{lstlisting}
		Residuals:
		Min        1Q    Median        3Q       Max 
		-0.052278 -0.017021  0.001822  0.012632  0.070010 
		
		Coefficients:
		Estimate Std. Error t value Pr(>|t|)    
		(Intercept)  0.05664    0.01157   4.894 6.52e-06 ***
		Adj          2.26526    0.16170  14.009  < 2e-16 ***
		---
		Residual standard error: 0.02366 on 67 degrees of freedom
		Multiple R-squared:  0.7455,	Adjusted R-squared:  0.7417 
		F-statistic: 196.2 on 1 and 67 DF,  p-value: < 2.2e-16
	\end{lstlisting}
\end{frame}
% ------------------------------------------------------------------------
\begin{frame}[t,fragile=singleslide]
\frametitle{Análise da correlação - negativa}
\vspace{0.5cm}
Teste de correlação: Ncomm vs P1 (negativa)
\vspace{0.5cm}
\begin{tikzpicture}[remember picture,overlay]
\node[xshift=8cm,yshift=0cm,opacity=1.0] at (current page.center) {\includegraphics[width=14cm]{../Rcodes/Scater_02.png}};
\end{tikzpicture}

\begin{lstlisting}
Pearson's product-moment correlation

data:  reg_bnc$Ncomm and reg_bnc$P1
t = -12.207, df = 67, p-value < 2.2e-16
alternative hypothesis: true correlation is not equal to 0
95 percent confidence interval:
-0.8919068 -0.7391773
sample estimates:
cor 
-0.8305535 
\end{lstlisting}
\end{frame}
% ------------------------------------------------------------------------
\begin{frame}[t,fragile=singleslide]
\frametitle{Análise da regressão - inclinação negativa}
\vspace{1cm}
Analisando o resultado da regressão
\vspace{0.5cm}
\begin{lstlisting}
Residuals:
Min        1Q    Median        3Q       Max 
-0.090100 -0.015763  0.001781  0.015173  0.044199 

Coefficients:
Estimate Std. Error t value Pr(>|t|)    
(Intercept)  0.263250   0.005128   51.33   <2e-16 ***
P1          -2.878204   0.235789  -12.21   <2e-16 ***
---
Residual standard error: 0.02611 on 67 degrees of freedom
Multiple R-squared:  0.6898,	Adjusted R-squared:  0.6852 
F-statistic:   149 on 1 and 67 DF,  p-value: < 2.2e-16
\end{lstlisting}
\end{frame}
% ------------------------------------------------------------------------
\begin{frame}[t,fragile=singleslide]
\frametitle{Análise da correlação}
\vspace{0.5cm}
Teste de correlação: Adj vs Nprop ($\approx$ zero)
\vspace{0.5cm}
\begin{tikzpicture}[remember picture,overlay]
\node[xshift=8cm,yshift=0cm,opacity=1.0] at (current page.center) {\includegraphics[width=14cm]{../Rcodes/Scater_03.png}};
\end{tikzpicture}

\begin{lstlisting}
Pearson's product-moment correlation

data:  reg_bnc$Adj and reg_bnc$Nprop
t = 1.0774, df = 67, p-value = 0.2852
alternative hypothesis: true correlation is not equal to 0
95 percent confidence interval:
-0.1095700  0.3561749
sample estimates:
cor 
0.1304949
\end{lstlisting}
\end{frame}
% ------------------------------------------------------------------------
\begin{frame}[t,fragile=singleslide]
\frametitle{Análise da regressão}
\vspace{1cm}
Analisando o resultado da regressão
\vspace{0.5cm}
\begin{lstlisting}
Residuals:
Min        1Q    Median        3Q       Max 
-0.035000 -0.013270  0.000802  0.012221  0.045236 

Coefficients:
Estimate Std. Error t value Pr(>|t|)    
(Intercept) 0.066167   0.003661  18.072   <2e-16 ***
Nprop       0.085743   0.079587   1.077    0.285    
---
Residual standard error: 0.01772 on 67 degrees of freedom
Multiple R-squared:  0.01703,	Adjusted R-squared:  0.002358 
F-statistic: 1.161 on 1 and 67 DF,  p-value: 0.2852
\end{lstlisting}
\end{frame}

% ------------------------------------------------------------------------
\begin{frame}[t,fragile=singleslide]
\frametitle{Fim de Regressão Linear e Correlação - Dever de casa}
\textbf{Exercícios do livro \cite{Agresti2018}:}
\begin{itemize}
	\item[-] Capítulo 9: 9.11, 9.18, 9.21, 9,24, 9.39.
\end{itemize}
\end{frame}
% ------------------------------------------------------------------------
\section{Relação Multivariável}
\begin{frame}[t,fragile=singleslide]
\frametitle{Variáveis informações e correlação}
\vspace{0.5cm}
\begin{itemize}
	\item[-] Associação (correlação) não implica causalidade (necessário as não suficiente).
	\item[-] Problema do controle estatístico nas ciências sociais.
	\item[-] Recorte de variáveis.
\end{itemize}
	\begin{tikzpicture}[remember picture,overlay]
		\node[xshift=0cm,yshift=-2.5cm,opacity=1.0] at (current page.center) {\includegraphics[width=16cm]{sharks_vs_icecream.png}};
	\end{tikzpicture}

\lfr{\url{https://www.nathalievialaneix.eu/teaching/biostat1/img/sharks_vs_icecream.png}}
\end{frame}
%% ------------------------------------------------------------------------
%\begin{frame}[t,fragile=singleslide]
%\frametitle{Formas de associação - Variável à espreita}
%\end{frame}
%% ------------------------------------------------------------------------
%\begin{frame}[t,fragile=singleslide]
%\frametitle{Formas de associação - Variável intermediária}
%\end{frame}
%% ------------------------------------------------------------------------
%\begin{frame}[t,fragile=singleslide]
%\frametitle{Formas de associação - Interação entre variáveis}
%\end{frame}
%% ------------------------------------------------------------------------
%\begin{frame}[t,fragile=singleslide]
%\frametitle{Formas de associação - Causa múltiplas}
%\end{frame}
%% ------------------------------------------------------------------------
%\begin{frame}[t,fragile=singleslide]
%\frametitle{Formas de associação - Efeitos Diretos e indiretos}
%\end{frame}
% ------------------------------------------------------------------------
\begin{frame}[t,fragile=singleslide]
\frametitle{Formas de associação - Resumo}

\begin{table}[]
	\centering
	\begin{tabularx}{31cm}{|C{6cm} L{9cm} L{13cm}|}
		\hline
		\rowcolor[HTML]{C0C0C0} Diagrama & Nome da relação                       & Efeito do controle de $x_2$                                                \\ \hline
		\includegraphics[height=2cm]{Cap_10_Espuria.png} 
		 & Associação espúria entre $x_1$ e $y$. & Associação entre $x_1$ e $y$ desaparece.                                   \\
		\includegraphics[height=1.25cm]{Cap_10_Cadeia.jpg} 
		& Relação em cadeia: $x_2$ intervêm, $x_1$ causa $y$ de forma indireta. & Associação entre $x_1$ e $y$ desaparece.                   \\
		\includegraphics[height=2cm]{Cap_10_Interacao.jpg} 
		& Interação.                            & Associação entre $x_1$ e $y$ varia de acordo com o nível (valor) de $x_2$. \\
		\includegraphics[height=2cm]{Cap_10_Causas_Multiplas.jpg} 
		& Causa múltiplas.                      & Associação entre $x_1$ e $y$ não se altera.                                \\
		\includegraphics[height=2cm]{Cap_10_Efeitos_Diretos_indiretos.jpg} 
		& Efeito direto e indireto de $x_1$ em $y$.                             & Associação entre $x_1$ e $y$ altera-se mas não desaparece. \\ \hline
	\end{tabularx}
\end{table}

\end{frame}
% ------------------------------------------------------------------------
\begin{frame}[t,fragile=singleslide]
\frametitle{Fim de Relação Multivariável - Dever de casa}
\textbf{Exercícios do livro \cite{Agresti2018}:}
\begin{itemize}
	\item[-] Capítulo 10: 10.5, 10.13, 10.19, 10.29.
\end{itemize}

\end{frame}
% ------------------------------------------------------------------------
\section{Regressão Múltipla e Correlação}
\begin{frame}[t,fragile=singleslide]
\frametitle{Dever de casa}
\end{frame}
% ------------------------------------------------------------------------
\begin{frame}[t,fragile=singleslide]
\frametitle{Dever de casa}
\end{frame}
% ------------------------------------------------------------------------
\section{Análise de Variância - ANOVA}
\begin{frame}[t,fragile=singleslide]
\frametitle{Dever de casa}
\end{frame}
% ------------------------------------------------------------------------
\begin{frame}[t,fragile=singleslide]
\frametitle{Dever de casa}
\end{frame}
% ------------------------------------------------------------------------
\section{Preditores Quantitativos e Categóricos}
\begin{frame}[t,fragile=singleslide]
\frametitle{Dever de casa}
\end{frame}
% ------------------------------------------------------------------------
\begin{frame}[t,fragile=singleslide]
\frametitle{Dever de casa}
\end{frame}
% ------------------------------------------------------------------------
\section{Modelos com Regressão Múltipla}
\begin{frame}[t,fragile=singleslide]
\frametitle{Dever de casa}
\end{frame}
% ------------------------------------------------------------------------
\begin{frame}[t,fragile=singleslide]
\frametitle{Dever de casa}
\end{frame}
% ------------------------------------------------------------------------
\section{Regressão Logística}
\begin{frame}[t,fragile=singleslide]
\frametitle{Dever de casa}
\end{frame}
% ------------------------------------------------------------------------
\begin{frame}[t,fragile=singleslide]
\frametitle{Dever de casa}
\end{frame}
% ------------------------------------------------------------------------
%\section{Introdução a métodos aprofundados}
%\begin{frame}[t,fragile=singleslide]
%\frametitle{Dever de casa}
%\end{frame}
%% ------------------------------------------------------------------------
%\begin{frame}[t,fragile=singleslide]
%\frametitle{Dever de casa}
%\end{frame}
% ------------------------------------------------------------------------
\section{Encerramento}
\begin{frame}[fragile=singleslide]
\frametitle{Sobre este material}%{Condições de uso e referência}

	Esta obra está licenciada sob a licença \textit{Creative Commons} CC BY-NC-SA 4.0 (mais detalhes neste \href{http://creativecommons.org/licenses/by-nc-sa/4.0/}{\textit{link}})\\
%	
	\flushleft
	Favor fazer referência a este trabalho como:\linebreak
	
	Silva, A. P. (2022), \textit{Notas de Aulas de Estatística para Linguística}. Online: {\url{https://github.com/adelinocpp/estatistica-para-linguistica}}
	\linebreak
	\begin{addmargin}[2cm]{0em}
		\normalsize 
		\begin{verbatim}
		@Misc{Silva2022,
		title={Notas de Aulas de Notas de Aulas de Estatística para Linguística},
		author={Adelino Pinheiro Silva},
		howPublished={\url{https://github.com/adelinocpp/estatistica-para-linguistica}},
		year={2022},
		note={Version 1.0; Creative Commons BY-NC-SA 4.0.},
		}
		\end{verbatim}
	\end{addmargin}
	\vfill
	\begin{tikzpicture} [remember picture,overlay]
		\node[anchor=south,yshift=0pt] at (current page.south){ \includegraphics[width=.1\textwidth]{00BAS_CCsomerights.png}};
	\end{tikzpicture}
\end{frame} 


% =======================================================================
\section{Referências}

\begin{frame}[allowframebreaks, t]{Referências}
	\bibliographystyle{apalike}
    \bibliography{EST_Aula_01_v00}
\end{frame}
% ======================================================================= 

\end{document}

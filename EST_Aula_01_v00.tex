\documentclass[14pt]{beamer}
\geometry{papersize={338.7mm,190.5mm},margin={1.125cm,0.125cm}}
\usepackage[utf8]{inputenc}
\usepackage[brazil]{babel}
\usetheme{boxes}
\usepackage[sfdefault]{carlito}
\usepackage{natbib}
% -----------------------------------------------------------------------
% --- Slide Justificado -------------------------------------------------
\usepackage{ragged2e}
\usepackage{etoolbox}
\usepackage{scrextend} % ajustar texto para direita
\usepackage{soul}
\usepackage[normalem]{ulem}
\usepackage{cancel}
\usepackage{bm}
\usepackage{array}
\usepackage{wrapfig} 
\apptocmd{\frame}{}{\justifying}{} % Allow optional arguments after frame.
\usepackage{amsmath,mathtools}
\usepackage{verbatim}
% -----------------------------------------------------------------------
% separação entre colunas e linhas de tabelas
\setlength{\tabcolsep}{2pt}
\renewcommand{\arraystretch}{1}
% -----------------------------------------------------------------------
% --- Define a centralização de coluna com largura {|C{1 cm}|}
\newcolumntype{C}[1]{>{\centering\let\newline\\\arraybackslash\hspace{0pt}}m{#1}}
\newcolumntype{L}[1]{>{\raggedright\let\newline\\\arraybackslash\hspace{0pt}}m{#1}}
% -----------------------------------------------------------------------
\newcommand\litem[1]{\item{\bfseries #1 }}

% --- Letras e simbolos matemáticos -------------------------------------
%\usepackage[bbgreekl]{mathbbol}  % Pacote para repreentação de conjuntos com \mathbb{R} com letras gregas
%\usepackage{amsfonts}	% Pacote para repreentação de conjuntos com \mathbb{R}
%\usepackage{mathrsfs}	% Pacote para letras matemáticas
%\usepackage{amssymb} 	% diversos simbolos matematicos adicionais. Carrega automático com amsfonts
% ------------------------------------------------------------------------
\usepackage{multicol}
\usepackage{multimedia}
\usepackage[]{graphicx}
\usepackage[]{color}
\usepackage{geometry}
%\usepackage{media9}

\usepackage{tabularx}
%\usepackage{amsmath, amsthm, amssymb}
\usepackage{gensymb}

% ------------------------------------------------------------------------

% --- Esta definicao deve vir antes --------------------------------------
\usepackage{graphicx}			% Inclusão de gráficos
\usepackage{float}  
% ------------------------------------------------------------------------
\usefonttheme[onlymath]{serif}
\usepackage{hyperref}
 \usepackage{breakurl}
\usepackage{multirow}
\usepackage{subfig}
\usepackage{ragged2e}
%\captionsetup[subfigure]{labelformat=empty}
\usepackage{color}
\usepackage{colortbl}
\usepackage{textpos}
\usepackage{tikz}
\usetikzlibrary{calc}
%---------------
\usepackage{tabularx}
\usepackage{booktabs}
\usepackage{multimedia}
\usepackage[style=british]{csquotes}
% ------------------------------------------------------------------------
\pdfsuppresswarningpagegroup=1
\def\disciplina/{Estatística para Linguística}
% ------------------------------------------------------------------------
\tikzset{
	invisible/.style={opacity=0},
	visible on/.style={alt={#1{}{invisible}}},
	alt/.code args={<#1>#2#3}{%
		\alt<#1>{\pgfkeysalso{#2}}{\pgfkeysalso{#3}} % \pgfkeysalso doesn't change the path
	},
}
% ------------------------------------------------------------------------
\def\signed #1{{\leavevmode\unskip\nobreak\hfil\penalty50\hskip1em
		\hbox{}\nobreak\hfill #1%
		\parfillskip=0pt \finalhyphendemerits=0 \endgraf}}
\newsavebox\mybox
\newenvironment{aquote}[1]
{\savebox\mybox{#1}\begin{quote}}
	{\vspace*{1mm}\signed{\usebox\mybox}\end{quote}}
% ------------------------------------------------------------------------
% --- Desativando os botoes de navegacao ---------------------------------
\setbeamertemplate{navigation symbols}{}
% ------------------------------------------------------------------------
% --- Tela cheia ---------------------------------------------------------
\hypersetup{pdfpagemode=FullScreen}
% ------------------------------------------------------------------------
% --- Layout da pagina ---------------------------------------------------
\hypersetup{pdfpagelayout=SinglePage}
% ------------------------------------------------------------------------
% --- Relaxed footnotes --------------------------------------------------
\newcommand{\lfr}[1]{\let\thefootnote\relax\footnote{\hspace{0.6cm}\vspace{1.25cm} #1}}
% --- Pasta com as imagens -----------------------------------------------
\graphicspath{{Imagens/}}
% --- Define o cinza tema ------------------------------------------------
\definecolor{pcmggray}{rgb}{0.5, 0.5, 0.5}
% --- Define o estilo do titulo do frame ---------------------------------
\setbeamertemplate{frametitle}{ 
	\Huge{\bfseries{\insertframetitle\par}\vskip-18pt\hrulefill}
	\begin{tikzpicture}[remember picture,overlay]
	\node[xshift=0.65cm,yshift=-1.28 cm,opacity=1.0] at (current page.north west) {\includegraphics[width=0.56cm]{00BAS_marcador_02.pdf}};
	\end{tikzpicture}
}
% ------------------------------------------------------------------------
\setbeamercolor{frametitle}{fg=pcmggray,bg=white}
\setbeamerfont{structure}{size=\LARGE}
\setbeamerfont{itemize/enumerate body}{size=\LARGE}
\setbeamerfont{itemize/enumerate subbody}{size=\LARGE}
\setbeamerfont{normal text}{size=\LARGE}
\setbeamercolor{structure}{fg=black} % itemize, enumerate, etc
\setbeamercolor{section in toc}{fg=black} % TOC sections
\setbeamertemplate{bibliography item}{}
\setbeamerfont{bibliography item}{size=\normalsize}
\setbeamerfont{bibliography entry author}{size=\normalsize}
\setbeamerfont{bibliography entry title}{size=\normalsize,series=\bfseries}
\setbeamerfont{bibliography entry location}{size=\normalsize}
\setbeamerfont{bibliography entry note}{size=\normalsize}
\AtBeginDocument{\usebeamerfont{normal text}}
% --- Fundo da pagina de titulo ------------------------------------------
\setbeamertemplate{background} 
{
	\begin{tikzpicture}[remember picture, overlay]
	\node[xshift=5.75cm,yshift=-1.68cm,opacity=1.0] at (current page.north west) {\includegraphics[width=11.49cm]{00BAS_titleleftupimage.pdf}};
	\node[xshift=16.93cm,yshift=-6.47cm,opacity=1.0] at (current page.north west) {\includegraphics[width=18.0cm]{00BAS_ada_ufmg.pdf}};
	\node[xshift=28.35cm,yshift=-17.48cm,opacity=1.0] at (current page.north west) {\includegraphics[width=11.04cm]{00BAS_titlerightdown.pdf}};
	\end{tikzpicture}
}
% ------------------------------------------------------------------------
\defbeamertemplate*{title page}{customized}[1][]
{
	\begin{center}
	\vspace{-1.5cm}
	\usebeamerfont{title}\LARGE{\textbf{\inserttitle}}\par
	\vspace{9cm}
	\usebeamerfont{subtitle}\huge{\textbf{\insertsubtitle}}\par
	\vspace{2cm}
	\usebeamerfont{author}\textbf{\insertauthor}\par
	\end{center}
	\vfill
}
\title{Laboratório de Fonologia}
\subtitle{\disciplina/}
\author[Silva, A. P.]{Prof. Dr. Adelino Pinheiro Silva}
\date{\today}
% =======================================================================
% INICIO DO CONTEUDO PRINCIPAL
% =======================================================================
\begin{document}
 \frame{\titlepage}
% =======================================================================
% ---- Fundo da demais paginas ------------------------------------------
\setbeamertemplate{background} 
{
	\begin{tikzpicture}[remember picture, overlay]
	\node[xshift=1.14cm,yshift=-18.08cm,opacity=1.0] at (current page.north west) {\includegraphics[width=1.58cm]{00BAS_fale_logo.pdf}};
	\node[xshift=28.97cm,yshift=-18.2cm,opacity=1.0] at (current page.north west) {\includegraphics[width=9.8cm]{00BAS_regularrightdown.pdf}};
	\end{tikzpicture}
}
% ==== Sumário ==========================================================
\section{Sumário}
\begin{frame}
    \frametitle{Título}
    \tableofcontents
\end{frame}
% ========================================================================

% ========================================================================
\AtBeginSection[]
{
  \begin{frame}
    \frametitle{Assuntos}
    \tableofcontents[currentsection]
  \end{frame}
}
% =======================================================================

% ==== AULA 01 ==========================================================
\section{Introdução}
% ------------------------------------------------------------------------
\begin{frame}
	\frametitle{In a hole in the ground there lived a...}

	Por que estudar estatística?
	\begin{itemize}
		\item[-] Compreender \textbf{fatores} que afetam um resultado.
		\item[-] Julgar de forma crítica as informações recebidas.
		\item[-] Argumentar estatisticamente.
	\end{itemize}
	
	\vspace{2cm}
		
	O que é estatística \citep{Agresti2018}?
	\begin{itemize}
		\item[-] Conjunto de métodos para se \textbf{obter} e \textbf{analisar} dados.
		\item[-] Metodologia baseada na \textbf{ocorrência} \\ para realizar \textbf{previsão}.
	\end{itemize}

	\begin{tikzpicture}[remember picture,overlay]
		\node[xshift=10.5cm,yshift=1.50cm,opacity=1.0] at (current page.center) {\includegraphics[width=12cm]{est_01_desvio_padrao.jpg}};
	\end{tikzpicture}
	\vfill
	\lfr{Imagem: \url{https://i.pinimg.com/originals/48/fb/95/48fb9570a8413fe83053df7c3599e7eb.jpg}}
\end{frame} 
% ------------------------------------------------------------------------
\begin{frame}
	\frametitle{I Have the High Ground}

	Alguns termos para começar
	\begin{itemize}
		\item[-] \textbf{Dado}: Observação obtida sobre o objeto de interesse.
		\item[-] \textbf{Observação}: Medida, ou informação coletada (sujeita a ruído e erros).
		\item[-] \textbf{Base de dados}: Conjunto de dados, e.g., \textit{general social survey}.
		\item[-] \textbf{População}: Conjunto total dos elementos (desconhecido, inacessível).
		\item[-] \textbf{Amostra}: subconjunto da população, dados (medidas) coletados.
		\item[-] \textbf{Parâmetro}: Fator (resumo) numérico da população (dica: letras gregas).
		\item[-] \textbf{Estatística}: Valor obtido da amostra !!!!!
		\item[-] \textbf{Ferramental}: R-studio
	\end{itemize}
	\begin{tikzpicture}[remember picture,overlay]
	\node[xshift=7.5cm,yshift=-4.0cm,opacity=1.0] at (current page.center) {\includegraphics[width=12cm]{R-studio-logo.png}};
	\end{tikzpicture}
	\vfill
	\lfr{Imagem: \url{https://www.python.org/static/community_logos/python-logo-generic.svg}}
\end{frame}  
% ------------------------------------------------------------------------
\begin{frame}
\frametitle{I Have the High Ground}
	\begin{tikzpicture}[remember picture,overlay]
	\node[xshift=0cm,yshift=0.0cm,opacity=1.0] at (current page.center) {\includegraphics[width=22cm]{amostra-populacao_v0.png}};
	\end{tikzpicture}
	\vfill
	\lfr{Imagem: \url{https://fernandafperes.com.br/intervalo-de-confianca/}}
\end{frame}
% ------------------------------------------------------------------------
\begin{frame}
\frametitle{Medida e amostra}
	Maneiras de extrair informações de interesse.\\
	\begin{itemize}
		\item[-] \textbf{Variável aleatória}: Característica que pode variar com os elementos da população ou amostra.
		
		\item[-] \textbf{Escala de medição}: Extensão onde a variável aleatória pode ser medida. Exemplos:
			\begin{itemize}
				\item[-] Categóricas: (cara, coroa), (derrota, empate, vitória); ou
				\item[-] Quantitativas: $\{ x \in \mathbb{R} | 0 \leq x \leq 1\} $, $[0,1]$
			\end{itemize}
	\end{itemize}


	Se caracteriza a variável aleatória como um resultado de uma experiência aleatória, que pode ser classificada como:
	\begin{itemize}
		\item[-] \textbf{Categóricas}: valores aceitos dentro de um limite de categorias (qualitativos?).
		\item[-] \textbf{Quantitativas}: valores numéricos de qualquer conjunto, e.g., $\mathbb{N}$, $\mathbb{R}$, $\mathbb{C}$
	\end{itemize}
	\vfill
	\lfr{Definição formal de variável aleatória: $X:\Omega \to \mathbb{R} \Leftrightarrow \{ \omega: X(\omega) \leq x \} \in \mathcal{F}, \forall x \in \mathbb{R} $}
\end{frame}

% ------------------------------------------------------------------------
\begin{frame}
\frametitle{Medida e amostra}
	Escalas:.\\
	\begin{itemize}
		\item[-] \textbf{Intervalar}: delimitação numérica.
		\item[-] \textbf{Nominal}: Nomes/categorias ``não ordenáveis'', e.g., preferência de cores;
		\item[-] \textbf{Ordenáveis}: Nomes/categorias que podem ser ordenadas em níveis, e.g., expectativa do curso (baixá, sem expectativa, alta).	
	\end{itemize}
	
	Detalhe: Em escalas categóricas é muito difícil garantir uma homogeneidade dos intervalos, i.e., se os intervalos das categorias possuem escalas de mesmo tamanho.
	
	

\end{frame}
% ------------------------------------------------------------------------
\begin{frame}
\frametitle{Variáveis estatísticas}
	\begin{columns}[T] % align columns
		\begin{column}{.40\textwidth}
			\begin{itemize}
				\item[-] \textbf{Amostra aleatória simples}: todas amostras de mesmo tamanho possem a mesma ``chance''. Seria um retrato da população(?).
				\item[-] \textbf{Métodos de amostragem, \textit{sample survey}}: Sistemática, estratificada, grupo (\textit{cluster}), multiestágios.
				\item[-] \textbf{Amostra enviesada}: alunos de uma sala de aula (?).
			\end{itemize}
		\end{column}%
		\hfill%
		\begin{column}{.48\textwidth}	
		
		\end{column}%
	\end{columns}
	\begin{tikzpicture}[remember picture,overlay]
	\node[xshift=7cm,yshift=0.0cm,opacity=1.0] at (current page.center) {\includegraphics[width=18cm]{variaveis-estatisticas.png}};
	\end{tikzpicture}
	\vfill
	\lfr{Adaptado de \url{https://artemetabolica.blogspot.com/2015/12/bioestatistica-1-variaveis.html}}
\end{frame}
% ------------------------------------------------------------------------
\begin{frame}
\frametitle{Estudo experimental}
	\textbf{Experimento:} Controlar variáveis independentes e observar a variação de variáveis dependentes para dar suporte ou refutar uma hipótese.\\
	\begin{itemize}
		\item[-] Compara ``tratamentos''.
		\item[-] Unidades de testes.
		\item[-] Grupos, pelo menos, ``controle'' e ``tratamento''.
		\item[-] Variáveis estranhas (predatórias).
	\end{itemize}
%	Efeitos do teste: principal e interativo\\
%	Regressão analítica
	Problemas experimentais\\
		\begin{itemize}
		\item[-] Variação do instrumento (ou pessoa que conduz parte dele).
		\item[-] Regressão analítica.
		\item[-] Viés de seleção.
		\item[-] Perda de unidade
	\end{itemize}	
	\begin{tikzpicture}[remember picture,overlay]
	\node[xshift=12cm,yshift=-4.0cm,opacity=1.0] at (current page.center) {\includegraphics[width=10cm]{Experimento-criancas.jpg}};
	\end{tikzpicture}

	\vfill
	\lfr{ \url{https://br.freepik.com/vetores-premium/}}
\end{frame}
% ------------------------------------------------------------------------
\begin{frame}
\frametitle{Experimento}
	\vspace{-1.5cm}
	Efeitos do teste: principal e interativo\\
	\vspace{1cm}
	\textbf{Soluções para experimentos}:\\
	\begin{itemize}
		\item[-] Aleatorização.
		\item[-] Emparelhamento.
		\item[-] controle estatístico.
		\item[-] Planejamento.
		\item[-] Medições a \textit{posteriori}.
	\end{itemize}
	\begin{tikzpicture}[remember picture,overlay]
	\node[xshift=12cm,yshift=-4.0cm,opacity=1.0] at (current page.center) {\includegraphics[width=10cm]{Experimento-criancas.jpg}};
	\end{tikzpicture}
	
	\vfill
	\lfr{ \url{https://br.freepik.com/vetores-premium/}}
	%	Efeitos do teste: pr
\end{frame}
% ------------------------------------------------------------------------
\begin{frame}
\frametitle{Estudo de Observação}
	\begin{tikzpicture}[remember picture,overlay]
\node[xshift=5cm,yshift=-2.5cm,opacity=1.0] at (current page.center) {\includegraphics[width=23cm]{exemplo-de-correlacao-sem-causalidade.png}};
\end{tikzpicture}
\vspace{-2.5cm}
	\begin{itemize}
		\item[-] Sem manipulação do objeto de estudo.
		\item[-] Grupos desbalanceados, difícil de realizar uma comparação adequada.
		\item[-] \textbf{Não permite estabelecer causa e efeito}.
		\item[-] Pode indicar uma relação \\ entre variáveis.
		\item[-] Uma variável não medida \\ pode ser responsável \\ pelo padrão observado.
	\end{itemize}
%	\begin{tikzpicture}[remember picture,overlay]
%	\node[xshift=12cm,yshift=-4.0cm,opacity=1.0] at (current page.center) {\includegraphics[width=12cm]{exemplo-de-correlação-sem-causalidade.jpg}};
%	\end{tikzpicture}

	\vfill
	\lfr{ \url{http://blog.bravi.com.br/wp-content/uploads/2017/10}}
\end{frame}
% ------------------------------------------------------------------------
\begin{frame}
\frametitle{Variabilidade amostral e viés}
\vspace{-2cm}
	\textbf{Erro de amostragem:} erro ocorrido ao utilizar uma estatística da amostra para predizer um parâmetro da população. Exemplo: Erro da pesquisa eleitoral com n = 100 de + ou - 3\%. \\
	\vspace{1cm}
	Viés: erro quando a amostra é enviesada, e.g., voluntários ou respostas de carta.\\
	\begin{itemize}
		\item[-] \textbf{Viés de resposta} ocorre quando a pergunta é confusa, e.g., referendo do desarmamento;
		\item[-] \textbf{viés de falha de dados} apenas uma fatia da amostra responde.
	\end{itemize}
	\vfill
\end{frame}
% ------------------------------------------------------------------------

	\begin{frame}
	\frametitle{Fim da introdução - Dever de casa}
		\textbf{Exercícios do livro \cite{Agresti2018}:}
		\begin{itemize}
			\item[-] Capítulo 1: 1.1, 1.3, 1.5-1.8, 1.14, 1.16;
			\item[-] Capítulo 2: 2.2-2.10,2.27, 2.35-2.37,2.39
		\end{itemize}
		\textbf{Preparação do terreno}
		\begin{itemize}
			\item[-] Instalar o R-studio.
		\end{itemize}	

	\end{frame}   
% ------------------------------------------------------------------------
\section{Estatística descritiva}
% ------------------------------------------------------------------------
\begin{frame}
\frametitle{Dever de casa}
\end{frame}
% ------------------------------------------------------------------------
\begin{frame}
\frametitle{Dever de casa}
\end{frame}
% ------------------------------------------------------------------------
\begin{frame}
\frametitle{Dever de casa}
\end{frame}
% ------------------------------------------------------------------------
\begin{frame}
\frametitle{Dever de casa}
\end{frame}
% ------------------------------------------------------------------------
\section{Encerramento}
\begin{frame}[fragile=singleslide]
\frametitle{Sobre este material}%{Condições de uso e referência}

	Esta obra está licenciada sob a licença \textit{Creative Commons} CC BY-NC-SA 4.0 (mais detalhes neste \href{http://creativecommons.org/licenses/by-nc-sa/4.0/}{\textit{link}})\\
%	
	\flushleft
	Favor fazer referência a este trabalho como:\linebreak
	
	Silva, A. P. (2022), \textit{Notas de Aulas de Estatística para Linguística}. Online: {\url{https://github.com/adelinocpp/}}
	\linebreak
	\begin{addmargin}[2cm]{0em}
		\normalsize 
		\begin{verbatim}
		@Misc{Silva2022,
		title={Notas de Aulas de Notas de Aulas de Estatística para Linguística},
		author={Adelino Pinheiro Silva},
		howPublished={\url{https://github.com/adelinocpp/}},
		year={2022},
		note={Version 1.0; Creative Commons BY-NC-SA 4.0.},
		}
		\end{verbatim}
	\end{addmargin}
	\vfill
	\begin{tikzpicture} [remember picture,overlay]
		\node[anchor=south,yshift=0pt] at (current page.south){ \includegraphics[width=.1\textwidth]{00BAS_CCsomerights.png}};
	\end{tikzpicture}
%
%\begin{lstlisting}
%bsbasas
%\end{lstlisting}
	

\end{frame} 


% =======================================================================
\section{Referências}

\begin{frame}[allowframebreaks, t]{Referências}
	\bibliographystyle{apalike}
    \bibliography{EST_Aula_01_v00}
\end{frame}
% ======================================================================= 

\end{document}
